\documentclass[preprint,12pt]{aastex}

\usepackage{color,hyperref}
\definecolor{linkcolor}{rgb}{0,0,0.5}
\hypersetup{colorlinks=true,linkcolor=linkcolor,citecolor=linkcolor,
            filecolor=linkcolor,urlcolor=linkcolor}

\usepackage{url}
\usepackage{algorithmic,algorithm}

\usepackage{listings}
\definecolor{lbcolor}{rgb}{0.9,0.9,0.9}
\lstset{%
    language=Python,
    basicstyle=\footnotesize\ttfamily,
    showspaces=false,
    showstringspaces=false,
    tabsize=2,
    breaklines=false,
    breakatwhitespace=true,
    identifierstyle=\ttfamily,
    keywordstyle=\bfseries\color[rgb]{0.133,0.545,0.133},
    commentstyle=\color[rgb]{0.133,0.545,0.133},
    stringstyle=\color[rgb]{0.627,0.126,0.941},
}

\usepackage{amssymb,amsmath}

\newcommand{\project}[1]{{\sffamily #1}}

\newcommand{\paper}{\emph{Article}}

\newcommand{\foreign}[1]{\emph{#1}}
\newcommand{\etal}{\foreign{et\,al.}}
\newcommand{\etc}{\foreign{etc.}}

\newcommand{\Fig}[1]{Figure~\ref{fig:#1}}
\newcommand{\fig}[1]{\Fig{#1}}
\newcommand{\figlabel}[1]{\label{fig:#1}}
\newcommand{\Tab}[1]{Table~\ref{tab:#1}}
\newcommand{\tab}[1]{\Tab{#1}}
\newcommand{\tablabel}[1]{\label{tab:#1}}
\newcommand{\Eq}[1]{Equation~(\ref{eq:#1})}
\newcommand{\eq}[1]{\Eq{#1}}
\newcommand{\eqlabel}[1]{\label{eq:#1}}
\newcommand{\Sect}[1]{Section~\ref{sect:#1}}
\newcommand{\sect}[1]{\Sect{#1}}
\newcommand{\App}[1]{Appendix~\ref{sect:#1}}
\newcommand{\app}[1]{\App{#1}}
\newcommand{\sectlabel}[1]{\label{sect:#1}}
\newcommand{\Algo}[1]{Algorithm~\ref{algo:#1}}
\newcommand{\algo}[1]{\Algo{#1}}
\newcommand{\algolabel}[1]{\label{algo:#1}}

% math symbols
\newcommand{\dd}{\ensuremath{\,\mathrm{d}}}
\newcommand{\bvec}[1]{\ensuremath{\boldsymbol{#1}}}
\newcommand{\unit}[1]{\ensuremath{\mathrm{#1}}}
\newcommand{\normal}[1]{\ensuremath{\mathcal{N}(#1)}}
\newcommand{\pr}[1]{\ensuremath{p (#1)}}
\newcommand{\setof}[1]{\ensuremath{\{ #1 \}}}


\newcommand{\kepler}{\project{Kepler}}

\begin{document}

\title{Sick transit likelihood function\ldots}

\newcommand{\nyu}{2}
\newcommand{\mpia}{3}
\author{%
    Daniel~Foreman-Mackey\altaffilmark{1,\nyu},
    David~W.~Hogg\altaffilmark{\nyu,\mpia},
    \etal
}
\altaffiltext{1}{To whom correspondence should be addressed:
                        \url{danfm@nyu.edu}}
\altaffiltext{\nyu}{Center for Cosmology and Particle Physics,
                        Department of Physics, New York University,
                        4 Washington Place, New York, NY, 10003, USA}
\altaffiltext{\mpia}{Max-Planck-Institut f\"ur Astronomie,
                        K\"onigstuhl 17, D-69117 Heidelberg, Germany}

\begin{abstract}



\end{abstract}

\keywords{%
exoplanets: sickness
---
code: open-source
---
keywords: made-up-by-Hogg
}

\section{Introduction}

The NASA \kepler\ Mission is a space telescope designed to detect
\emph{Earth-like} transiting planets orbiting---on close to year-long
periods---around \emph{Sun-like} stars \citep{kepler}.
The photometric precision of the instrument is unprecedented (CITE) and
careful mining analysis of the data stream has lead to many impressive
discoveries (CITE).
While several nearly Earth-sized planets have been discovered recently (which
ones?), these signals are at the edge of the instrument's capabilities and
there are systematic signals---both astrophysical and instrumental---that are
of much larger magnitude than the transit signature.
As a result, one must carefully model the systematic effects as well as the
planetary signals.
In the standard lore, this procedure involves a ``de-trending'' step followed
by a ``modeling'' step.


\begin{itemize}

\item{describe Kepler data and systematics}
\item{describe and discuss problems with standard practices}
\item{\citet{gibson-gp}}

\end{itemize}

\section{Data}

\begin{itemize}

\item{describe the issues with the Kepler data in more detail}
\item{justify the fact that most of the data is not useful for constraining
the planet parameters}
\item{use this to justify throwing away most of the data except around the
transits}

\end{itemize}

\section{Model}

\begin{itemize}

\item{briefly sketch GP math and refer interested readers to
\citet{bishop-book} and \citet{gibson-gp}}
\item{justify a simple GP model qualitative by including a plot of data near
transits in a quiet star}
\item{discuss the fact that the model can actually be much more flexible and
can include components for variable stars and stellar activity, etc.}
\item{mention Bart as in prep}

\end{itemize}

\section{Implementation}

Discuss (briefly) implementation details and gotchas.

\section{Search}

\begin{itemize}

\item{scaling of error, etc.\ with decreasing dataset size + binning}
\item{try searching some light curves with injected transits and some without}
\item{automated vetting?}

\end{itemize}

\section{Inferred Planet Radius}

\begin{itemize}

\item{show that the standard methods underestimate planet radius (fingers
crossed) using injected transits}
\item{show that this method works better and run on a few interesting cases}

\end{itemize}

\section{Photoeccentricity, TTVs, etc.}

We should just be better for all of these things.
Can I show that?


\acknowledgments
SAMSI. %
It is a pleasure to thank
    Fengji Hou (NYU),
for helpful contributions to the ideas and code presented here.
This project was partially supported by the NSF (grant AST-0908357), and NASA
(grant NNX08AJ48G).

\newcommand{\arxiv}[1]{\href{http://arxiv.org/abs/#1}{arXiv:#1}}
\begin{thebibliography}{}\raggedright

\bibitem[Bishop(2003)]{bishop-book}
Bishop, C.~M., \emph{Pattern Recognition and Machine Learning}, Springer, 2009

\bibitem[Gibson \etal(2012)]{gibson-gp}
Gibson, N.~P., Aigrain, S., Roberts, S., \etal\ 2012, \mnras, 419, 2683

\bibitem[Koch \etal(2010)]{kepler}
Koch, D.~G., Borucki, W.~J., Basri, G., \etal\ 2010, \apjl, 713, L79

\end{thebibliography}

\end{document}
