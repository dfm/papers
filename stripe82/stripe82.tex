% Work in progress
% (c) Daniel Foreman-Mackey, all rights reserved

\documentclass[12pt,preprint]{aastex}
\usepackage{amssymb,amsmath,color}

% code listings setup
\usepackage{listings}
\lstset{ %
language=SQL,
basicstyle=\ttfamily\scriptsize,
tabsize=4,
keywordstyle=\bfseries,
identifierstyle=\bfseries,
frame=single,
}


% Specialized text styles
\newcommand{\project}[1]{\textsl{#1}}
\newcommand{\SDSS}{\project{SDSS}}
\newcommand{\foreign}[1]{\emph{#1}}
\newcommand{\etal}{\foreign{et\,al.}}
\newcommand{\etc}{\foreign{etc.}}

% Convenience functions for referencing equations, figures, etc.
% NOTE: the reference key is automatically prefixed with fig:, eq:, etc.
\newcommand{\figlabel}[1]{\label{fig:#1}}
\newcommand{\Fig}[1]{Figure \ref{fig:#1}}
\newcommand{\fig}[1]{figure \ref{fig:#1}}
\newcommand{\eqlabel}[1]{\label{eq:#1}}
\renewcommand{\eqref}[1]{\ref{eq:#1}}
\newcommand{\Eq}[1]{Equation (\eqref{#1})}
\newcommand{\eq}[1]{equation (\eqref{#1})}

% math symbols
\newcommand{\dd}{\mathrm{d}}
\newcommand{\like}{\mathscr{L}}
\newcommand{\paramvector}[1]{\boldsymbol{#1}}
\newcommand{\bvec}[1]{\paramvector{#1}}
\newcommand{\normal}[3]{\mathcal{N} (#1 | #2, #3)}
\newcommand{\grad}[2]{\frac{\partial #1}{\partial #2}}
\renewcommand{\Pr}{\mathrm{p}}

% model parameters
\newcommand{\model}     {\ensuremath{\paramvector{\Theta}} }
\newcommand{\data}      {\ensuremath{\mathbf{X}} }
\newcommand{\X}         {\ensuremath{\mathbf{X}} }
\newcommand{\fstar}     {\ensuremath{f^*} }
\newcommand{\fstarvec}  {\ensuremath{\mathbf{\fstar}} }
\newcommand{\fzero}     {\ensuremath{f^0}}
\newcommand{\fzerovec}  {\ensuremath{\mathbf{\fzero}} }
\newcommand{\jabs}      {\ensuremath{\delta} }
\newcommand{\jrel}      {\ensuremath{\eta} }
\newcommand{\jitter}[1] {\ensuremath{\delta_{#1}^2} }
\newcommand{\Stot}      {\ensuremath{\sigma_{i\alpha}^2 + \jitter{i\alpha}} }
\newcommand{\Sbad}      {\ensuremath{\Sigma_\mathrm{bad}^2} }
\newcommand{\Sbadtot}   {\ensuremath{\sigma_{i\alpha}^2 + \jitter{i\alpha} + \Sbad} }
\newcommand{\Qbad}      {\ensuremath{Q_\mathrm{bad}} }
\newcommand{\Qvar}      {\ensuremath{Q_\mathrm{var}} }
\newcommand{\Svar}      {\ensuremath{S} }
\newcommand{\Svartot}   {\ensuremath{\sigma_{i\alpha}^2 + \jitter{i\alpha} + \Svar} }
\newcommand{\oddsvar}   {\ensuremath{r^\mathrm{var}} }
\newcommand{\oddsbad}   {\ensuremath{r^\mathrm{bad}} }
\newcommand{\good}      {\ensuremath{H_\mathrm{g}}}
\newcommand{\bad}       {\ensuremath{H_\mathrm{b}}}
\newcommand{\const}     {\ensuremath{H_\mathrm{c}}}
\newcommand{\var}       {\ensuremath{H_\mathrm{v}}}
\newcommand{\pconst}    {\ensuremath{\Pr_\mathrm{c} (\X_\alpha | \model) }}
\newcommand{\pvar}      {\ensuremath{\Pr_\mathrm{v} (\X_\alpha | \model) }}
\newcommand{\pbad}      {\ensuremath{\Pr_\mathrm{b} (\X_{i\alpha} | \model)} }
\newcommand{\pvarbad}   {\ensuremath{\Pr_\mathrm{v,b} (\X_{i\alpha} | \model)} }
\newcommand{\pvargood}  {\ensuremath{\Pr_\mathrm{v,g} (\X_{i\alpha} | \model)} }
\newcommand{\pconstbad} {\ensuremath{\Pr_\mathrm{c,b} (\X_{i\alpha} | \model)} }
\newcommand{\pconstgood}{\ensuremath{\Pr_\mathrm{c,g} (\X_{i\alpha} | \model)} }

% units
\newcommand{\unit}[1]{\,\mathrm{#1}}

\begin{document}

\title{Recalibration of SDSS Stripe 82}
\author{Daniel~Foreman-Mackey\altaffilmark{1,2} \&
        David~W.~Hogg\altaffilmark{1,3}, others}
\altaffiltext{1}{Center for Cosmology, Particle Physics,
                         Department of Physics, New York University,
                         4 Washington Place, New York, NY, 10003, USA}
\altaffiltext{2}{to whom correspondence should be addressed:
                 danfm@nyu.edu}
\altaffiltext{3}{Max-Planck-Institut f\"ur Astronomie,
                 K\"onigstuhl 17, D-69117, Heidelberg, Germany}

\begin{abstract}

We develop a general probabilistic framework for the self-calibration of multi-
epoch astronomical imaging.  This model quantitatively accounts for variable stars,
bad (or missing) observations and underestimated uncertainties.  Using this framework,
we re-calibrate Stripe 82 from the Sloan Digital Sky Survey (\SDSS).  Stripe 82 is
a $\sim 300$ square-degree patch that was imaged $\gtrsim 50$ times over a 10 year
baseline.  Many runs were taken in non-photometric conditions with non-trivial
spatial and time varying conditions.  We define photometricity criteria based on
the scatter in the light curves of a set of standard stars and the width of the
stellar locus (DFM: be more specific?).  Using this model we re-calibrate the
photometry in Stripe 82 to better than $2\%$ accuracy in all 5 bands (???) and
recover X\% of the officially non-photometric runs.
We make the calibration information through a public interface along with the
light-curves for {\em some huge number} of stars.

\end{abstract}

\keywords{
}

\section{Introduction}

\section{Re-calibration}

\section{Queries}

\lstinputlisting[float=tp,caption=The fields query]{queries/fields.sql}
\lstinputlisting[float=tp,caption=The calibration star query]{queries/stars.sql}
\lstinputlisting[float=tp,caption=The Lyrae candidate query]{queries/lyrae.sql}


%To calibrate the photometric zero-point of each field where a particular RR
%Lyrae candidate was observed, we select (from CAS) all stars with $g < 20$
%within a radius $3^\prime$ (Note: these cuts are up for discussion), perform
%the following analysis.  In what follows, this is the sample of stars that we
%will discuss.

\subsection{Calibration Model}

(Note: Roman indices will indicate the epoch number, Greek indices will
label the star.)

We measure the flux of a star $\alpha$ at epoch $i$, $C_{i\alpha}$ with
variance $\sigma_{i\alpha}^2$.  Given these measurements of $N$ stars observed
at $M$ epochs, we wish to construct a probability distribution for the data given
parameters describing the intrinsic flux of each star, the zero-point of each
observation.

% The set of observations all epochs for star $\alpha$ is called
% $\bvec{D}_\alpha$, the set of all observations at a single epoch $i$ is
% $\bvec{D}_i$.

The simplest model of the full dataset $\data \equiv \{ C_{i\alpha},
\sigma^2_{i\alpha} \}$ includes an intrinsic flux $\fstar_\alpha$ for
each star (assumed constant at all epochs), a multiplicative zero-point
$\fzero_i$ (assumed constant across a given field).  Therefore,
$\bvec{\fstar}$ is a vector of length $N$, $\bvec{\fzero}$ has
length $M$, the model generates a proposal matrix ${\tilde{C}_{i\alpha}}
\equiv \bvec{\fzero} \otimes \bvec{\fstar}^T$ which is directly
compared to the data $C_{i\alpha}$ assuming Gaussian uncertainties.  This
yields the likelihood function for the full set of data, given a model
$\model$:
\begin{equation}
    \Pr (\data | \model) = \prod_{i=1} ^{M}
        \prod_{\alpha=1} ^{N} \normal{C_{i\alpha}}{\fzero_i
        \fstar_\alpha}{\sigma_{i\alpha}^2}
\end{equation}

It is likely that the measured uncertainties are underestimated, therefore, we
can also include ``jitter'' added to the quoted variance.  This allows the true
uncertainty to be inferred from the data even if the quoted value is
underestimated.
\begin{equation} \eqlabel{calib-likelihood2}
    \Pr (\data | \model) = \prod_{i=1}^{M} \prod_{\alpha=1} ^{N}
        \normal{C_{i\alpha}}{\fzero_i
        \fstar_\alpha}{\sigma_{i\alpha}^2 + \jitter{i\alpha}}
\end{equation}
where
\begin{equation}
    \jitter{i\alpha} \equiv \jabs^2 + (\jrel \, \fzero_{i}
        \fstar_\alpha)^2
\end{equation}
is the jitter, $\jabs$ and $\jrel$ are model parameters.

To generalize the model to include the possibility that not all stars are
useful for calibration (perhaps they vary intrinsically), we can generate a
``mixture model'' containing the distribution from equation
\eq{calib-likelihood2}, a general background distribution with variance
$\Svar$.  The background model is included with a probability $\Qvar$:
\begin{equation}
    \eqlabel{fulllikelihood}
    \Pr (\data | \model) = \prod_{\alpha=1} ^{N} \left \{ [1-\Qvar]
        \pconst + \Qvar
        \pvar \right \}
\end{equation}
where
\begin{equation}
    \pconst \equiv \Pr(\data | \model, \const) \, \mathrm{and} \,
    \pvar \equiv \Pr(\data | \model, \var)
\end{equation}
are the likelihoods of the data for star $\alpha$ given the model parameters and
the hypothesis that the star is constant (\const) or variable (\var) respectively.
\Eq{fulllikelihood} gives the likelihood {\em marginalized}
over the choice of \var\ or \const\ for each star \citep[e.g.][]{Hogg:2010}.

Since some observations
of a particular star might not be useful for calibration, we model
$\pconst$ and $\pvar$ as mixtures with background
populations of ``bad'' observations included with the prior probability $\Qbad$:
\begin{equation}
    \pconst \equiv \prod_{i=1}^M \left \{
    [1-\Qbad] \pconstgood + \Qbad \pconstbad \right \}
\end{equation}
and
\begin{equation}
    \pvar \equiv \prod_{i=1}^M \left \{
    [1-\Qbad] \pvargood + \Qbad \pvarbad \right \}
\end{equation}
where
\begin{eqnarray}
    \pconstgood & \equiv &
        \normal{C_{i\alpha}}{\fzero_i \fstar_\alpha}{\sigma_{i\alpha}^2 +
            \jitter{i\alpha}} \\
    \pconstbad \equiv \pvarbad \equiv \pbad & \equiv &
        \normal{C_{i\alpha}}{\fzero_i \fstar_\alpha}{\sigma_{i\alpha}^2 +
        \jitter{i\alpha} + \Sbad} \\
    \pvargood & \equiv &
        \normal{C_{i\alpha}}{\fzero_i \fstar_\alpha}{\sigma_{i\alpha}^2 +
        \jitter{i\alpha} + [\Svar \fzero_i \fstar_\alpha]^2} \\
\end{eqnarray}
are assumed Gaussian.

To summarize, this model has $N+M+6$ parameters:
\begin{equation}
    \model = \{ \fstarvec, \fzerovec, \jabs, \jrel, \Sbad, \Qbad, \Svar, \Qvar \}
\end{equation}
where \fstarvec, \fzerovec are intrinsic properties of the stars and runs
respectively, \jabs, \jrel describe the noise model for the flux measurements,
\Sbad, \Qbad parameterize the background ``bad measurement'' model and \Svar,
\Qvar specify the variability model.  In practice, the values of the six nuisance
parameters are inferred by finding the maximum likelihood (or maximum
\foreign{a posteriori}) values.


\subsection{Odds Ratios}

This model also yields the probability that a specific star is intrinsically
variable, that a given measurement is bad.  These quantities are conveniently
quantified by the odds ratios
\begin{equation}
    \oddsvar_\alpha \equiv \frac{\Qvar \pvar}{[1-\Qvar] \pconst}
\end{equation}
and
\begin{eqnarray}
    \oddsbad_{i\alpha} & \equiv & \frac{\Qbad}{1-\Qbad}
        \frac{[1-\Qvar] \pconst \pconstbad + \Qvar \pvar \pvarbad}
             {[1-\Qvar] \pconst \pconstgood+ \Qvar \pvar \pvargood} \\
    & = & \frac{\Qbad}{1-\Qbad} \frac{\pbad \{ [1-\Qvar] \pconst + \Qvar \pvar \}}
             {[1-\Qvar] \pconst \pconstgood+ \Qvar \pvar \pvargood}
\end{eqnarray}
Both $\oddsvar$, $\oddsbad$ will be in the range $ [0,\infty ) $ with larger
values indicating that the star is more likely to be variable or the measurement is
more likely to be bad (respectively).


\subsection{Constraints From Co-Added Analysis}

Up to this point, our photometric model, while completely self-consistent, will
not measure magnitudes that are meaningful outside the model.  It is reasonable
to require that our results are consistent with the magnitudes measured in
the co-added photometric catalogs provided by the \SDSS\ Catalog Archive Server
(CAS).  To achieve this, we include a further on the magnitude produced by the
model by multiplying \eq{modelprob} by
% TODO: Justify the choice of 0.5mag for scatter in this constraint
\begin{equation}
    p_\mathrm{CAS} (\data | \model) = \normal{m_{g,\alpha}}{m^*_\alpha}{0.5 \unit{mag}}
\end{equation}
where $m_{g,\alpha}$ is the $g$-band magnitude from CAS, $m^*_\alpha
\equiv -2.5\log_{10} \fstar_\alpha$ is the apparent magnitude generated
by the model.



\subsection{Extending the Model Across the Full Dataset}

The zero point of the \SDSS~camera should only vary slowly, smoothly with
time as the camera scans across the sky.  In the \SDSS~database, these coherent
datasets are called ``runs'', they sweep in right ascension at an
approximately constant declination.  We also split the data by ``camera column''.
We then optimize the above model for stars found in a projected radius $R$ around
for a grid of points with spacing $\Delta \theta$ in R.A.~along the length of
a run, interpolate between points using a cubic spline.

To determine the optimal values of $R$, $\Delta \theta$, we find the values
that maximize the leave-one-out cross-validation likelihood.  The cross-validation
likelihood is calculated by dividing the data points into $N$ groups and
calculating the likelihood of observing one group given a model optimized for
the set of the $N-1$ other groups. In practice, we use the mean likelihood over
all $N \equiv 10$ iterations of this technique for a grid in the space of $R$
and $\Delta \theta$.



\bibliographystyle{apj}

\begin{thebibliography}{90}

\bibitem[Hogg \etal(2010))]{Hogg:2010}
{Hogg}, D.~W., {Bovy}, J., \& {Lang}, D.,
2010, arXiv:1008.4686
% http://adsabs.harvard.edu/abs/2010arXiv1008.4686H

\bibitem[Sesar \etal(2010)]{Sesar:2010}
{Sesar}, B., {Ivezic}, Z., {Grammer}, S.~H., {Morgan}, D.~P., {Becker}, A.~C.,
{Jurić}, M., {De Lee}, N., {Annis}, J., {Beers}, T.~C., {Fan}, X.,
{Lupton}, R.~H., {Gunn}, J.~E., {Knapp}, G.~R., {Jiang}, L., {Jester}, S.,
{Johnston}, D.~E., {Lampeitl}, H.,
2010, \apj, 708, 717
% http://adsabs.harvard.edu/abs/2010ApJ...708..717S

\end{thebibliography}

\appendix

\section{Gradients of the Likelihood Function}

The gradient of \eq{fulllikelihood} with respect to \model can be easily
calculated as follows.

\subsection{Gradients with respect to physical quantities}

The gradient of the likelihood function with respect to the (modelled)
{\bf intrinsic stellar flux} is calculated using
\begin{equation}
    \grad{\ln \Pr (\data | \model)}{\fstar_\alpha} =
        \frac{(1-\Qvar)\grad{\pconst}{\fstar_\alpha} + \Qvar \grad{\pvar}{\fstar_\alpha}}
            { (1-\Qvar)\pconst + \Qvar\pvar}
\end{equation}
where
\begin{equation} \eqlabel{dpconstdalpha}
    \grad{\pconst}{\fstar_\alpha} = \pconst \grad{\ln \pconst}{\fstar_\alpha}
\end{equation}
and
\begin{equation}
    \grad{\ln \pconst}{\fstar_\alpha} =
        \sum_{i = 1}^{M}
            \frac{(1-\Qbad)\grad{\pconstgood}{\fstar_\alpha}
                + \Qbad \grad{\pconstbad}{\fstar_\alpha}}{ (1-\Qbad)
            \pconstgood  + \Qbad
            \pconstbad}.
\end{equation}
The derivatives in this equation are simply derivatives of the respective normal
distributions:
\begin{equation} \eqlabel{dpconstgooddalpha}
    \grad{\pconstgood}{\fstar_\alpha} = \fzero_i \left [
        \frac{C_{i\alpha} - \fstar_\alpha \fzero_i}{\Stot} \right ]
        \pconstgood
\end{equation}
and
\begin{equation} \eqlabel{dpconstbaddalpha}
    \grad{\pconstbad}{\fstar_\alpha} = \fzero_i \left [
        \frac{C_{i\alpha} - \fstar_\alpha \fzero_i}{\Sbadtot} \right ]
        \pconstbad.
\end{equation}
In analogy to equations \eqref{dpconstdalpha} through \eqref{dpconstbaddalpha},
we can calculate $\grad{\pvar}{\fstar_\alpha}$ by replacing the variance in the
denominator of \eq{dpconstgooddalpha} by \Svartot.

Now, we find the gradient of the likelihood with respect to the {\bf photometric
zero point}:
\begin{equation}
    \grad{\ln \Pr (\data | \model)}{\fzero_i} = \sum_{\alpha=1}^{N}
        \frac{(1-\Qvar)\grad{\pconst}{\fzero_i} + \Qvar \grad{\pvar}{\fzero_i}}
            { (1-\Qvar)\pconst + \Qvar\pvar}
\end{equation}
where
\begin{equation}
    \grad{\pconst}{\fzero_i} = \pconst \times
            \frac{(1-\Qbad)\grad{\pconstgood}{\fzero_i}
                + \Qbad \grad{\pconstbad}{\fzero_i}}{ (1-\Qbad)
            \pconstgood  + \Qbad
            \pconstbad}.
\end{equation}
The derivatives of the normal distributions will be equations \eqref{dpconstgooddalpha}
and \eqref{dpconstbaddalpha} above with $\fstar_\alpha$, $\fzero_i$ swapped.


\subsection{Gradients with respect to the nuisance parameters}

First, the gradient of the likelihood function with respect to the {\bf prior
probability of a star being variable} is simply
\begin{equation}
    \grad{\ln \Pr (\data | \model)}{\Qvar} =
        \sum_{\alpha = 1}^{N} \frac{\pvar - \pconst}{ (1-\Qvar)
            \pconst  + \Qvar
            \pvar}
\end{equation}

The gradient with respect to the {\bf prior probability that an observation is
bad} is
\begin{equation}
    \grad{\log \Pr (\data | \model)}{\Qbad} =
        \sum_{\alpha = 1}^{N} \frac{(1-\Qvar)
            \grad{\pconst}{\Qbad}  + \Qvar
            \grad{\pvar}{\Qbad}}{ (1-\Qvar) \pconst  + \Qvar \pconst}
\end{equation}
where
\begin{equation}
    \grad{\ln \pconst}{\Qbad} = \pconst \times
        \sum_{i = 1}^{M} \frac{\pconstbad - \pconstgood}{ (1-\Qbad)
            \pconstgood  + \Qbad
            \pconstbad}.
\end{equation}
Similarly,
\begin{equation}
    \grad{\pvar}{\Qbad} = \pvar \times
        \sum_{i = 1}^{M} \frac{\pvarbad - \pvargood}{ (1-\Qbad)
            \pvargood  + \Qbad
            \pvarbad}.
\end{equation}

Finally, the gradients with respect to the model variances ($V^2 = \jrel^2, \jabs^2, \Svar, \Sbad$)
are calculated as
\begin{equation}
    \grad{\ln \Pr (\data | \model)}{V^2} = \sum_{\alpha=1}^{N}
        \frac{(1-\Qvar)\grad{\pconst}{V^2} + \Qvar \grad{\pvar}{V^2}}
            { (1-\Qvar)\pconst + \Qvar\pvar}
\end{equation}
where
\begin{equation}
    \grad{\ln \pconst}{V^2} =
        \sum_{i = 1}^{M}
            \frac{(1-\Qbad)\grad{\pconstgood}{V^2}
                + \Qbad \grad{\pconstbad}{V^2}}{ (1-\Qbad)
            \pconstgood  + \Qbad
            \pconstbad}.
\end{equation}
and
\begin{equation}
    \grad{\pconstgood}{V^2} = \grad{(\Stot)}{V^2} \times
        \left [ \frac{(C_{i\alpha} - \fstar_\alpha \fzero_i)^2-(\Stot)}{2(\Stot)} \right ]
        \times \pconstgood,
\end{equation}
\foreign{etc.}




% =========== %
%   FIGURES   %
% =========== %

\clearpage

\begin{figure}
    \begin{center}
        \includegraphics[width=0.8\textwidth,trim=1.9cm 5cm 2.5cm 3cm,clip=true]{gm/gm.pdf}
    \end{center}
    \caption{Our sick probabilistic graphical model.
    \figlabel{gm}}
\end{figure}

\end{document}


