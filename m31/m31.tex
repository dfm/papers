% Work in progress
%   (c) Daniel Foreman-Mackey, all rights reserved


% TODO
% ----
%

\documentclass[preprint]{aastex}

\usepackage{amssymb,amsmath}

\usepackage{color,hyperref}
\definecolor{linkcolor}{rgb}{0,0,0.5}
\hypersetup{
    colorlinks=true,
    linkcolor =linkcolor,
    citecolor =linkcolor,
    filecolor =linkcolor,
    urlcolor  =linkcolor
}

\usepackage{multirow}

%% My definitions
\newcommand{\foreign}[1]{\emph{#1}}
\newcommand{\etal}{\foreign{et\,al.}}
\newcommand{\eg}{\foreign{e.g.}}
\newcommand{\etc}{\foreign{etc.}}

\newcommand{\project}[1]{\emph{#1}}
\newcommand{\sdss}{\project{SDSS}}
\newcommand{\pandas}{\project{PAndAS}}

\newcommand{\andromeda}{\object[M31]{Andromeda}}

% Convenience functions for referencing equations, figures, etc.
% NOTE: the reference key is automatically prefixed with fig:, eq:, etc.
\newcommand{\figlabel}[1]{\label{fig:#1}}
\newcommand{\Fig}[1]{Figure \ref{fig:#1}}
\newcommand{\fig}[1]{figure \ref{fig:#1}}
\newcommand{\tablabel}[1]{\label{tab:#1}}
\newcommand{\Tab}[1]{Table \ref{tab:#1}}
\newcommand{\tab}[1]{table \ref{tab:#1}}
\newcommand{\eqlabel}[1]{\label{eq:#1}}
\newcommand{\Eq}[1]{Equation \ref{eq:#1}}
\newcommand{\eq}[1]{equation \ref{eq:#1}}

% math symbols
\newcommand{\dd}{\mathrm{d}}
\newcommand{\bvec}[1]{\ensuremath{\boldsymbol{#1}}}
\newcommand{\paramvector}[1]{\bvec{#1}}
\newcommand{\normal}[3]{\ensuremath{\mathcal{N} (#1 | #2, #3)}}

% Bayesianism
\newcommand{\model}{\paramvector{\Theta}}
\newcommand{\data}{\paramvector{X}}
\newcommand{\prob}{\ensuremath{p}}
\newcommand{\probability}[2]{\ensuremath{\prob ( #1 | #2 )}}
\newcommand{\likelihood}{\probability{\data}{\model}}
\newcommand{\posterior}{\probability{\model}{\data}}
\newcommand{\prior}{\ensuremath{\prob (\model)}}

% distribution functions
\newcommand{\df}{f}
\newcommand{\dfs}[1]{\df_\mathrm{#1}}
\newcommand{\dfhalo}{\dfs{halo}}
\newcommand{\dfbulge}{\dfs{bulge}}
\newcommand{\dfdisk}{\dfs{disk}}

\newcommand{\E}{\ensuremath{E}}
\newcommand{\Lz}{\ensuremath{L_Z}}
\newcommand{\Ez}{\ensuremath{E_Z}}

% model parameters
\newcommand{\halo}{\mathrm{h}}
\newcommand{\bulge}{\mathrm{b}}
\newcommand{\disk}{\mathrm{d}}
\newcommand{\mass}{\ensuremath{M}}

\newcommand{\ah}{\ensuremath{a_\halo}}
\newcommand{\vh}{\ensuremath{\sigma_\halo}}
\newcommand{\nh}{\ensuremath{n_\halo}}
\newcommand{\Md}{\ensuremath{\mass_\disk}}
\newcommand{\Rd}{\ensuremath{a_\disk}}
\newcommand{\vb}{\ensuremath{\sigma_\bulge}}
\newcommand{\Rb}{\ensuremath{a_\bulge}}
\newcommand{\nb}{\ensuremath{n_\bulge}}

% velocity corrections
\newcommand{\vlos}{\ensuremath{v_\mathrm{los}}}
\newcommand{\vt}{\ensuremath{v_\mathrm{t}}}
\newcommand{\vrel}[2]{\ensuremath{\bvec{v}_{\mathrm{#1},\mathrm{#2}}}}

% random other stuff
\newcommand{\Rband}{\emph{R}-band}
\newcommand{\Kband}{\emph{R}-band}

% units
\newcommand{\unit}[1]{\,\mathrm{#1}}
\newcommand{\kpc}{\unit{kpc}}

\begin{document}

\title{The Mass of \andromeda: A Fully Self-Consistent Dynamical Model}

% \slugcomment{To be submitted for publication in ApJ}

\author{Daniel Foreman-Mackey\altaffilmark{\ref{affil:dfm}}
    \& Lawrence M. Widrow}
\affil{Department of Physics, Engineering Physics, and Astronomy,
       Queen's University, Kingston, ON K7L 3N6, Canada}

\altaffiltext{1}{\label{affil:dfm}Current address:
    Center for Cosmology and Particle Physics, Department of Physics,
    New York University,
    4 Washington Place, New York, NY 10003, USA; \texttt{danfm@nyu.edu}}

\begin{abstract}

    Due to it's proximity, there is a wealth of kinematic measurements of M31
    at all scales. These datasets provide an extremely detailed snapshot of the
    dynamics of the galaxy at both dark matter and baryon dominated scales. We
    compile a catalog of satellite galaxy kinematics in the outer reaches of the
    dark matter halo and the HI rotation curve and (K-, I-band) surface brightness
    profile at smaller radii.
    To combine these heterogeneous datasets into a self-consistent probabilistic
    analysis, we develop a multicomponent, axisymmetric, semi-analytic, dynamical
    generative galaxy model and marginalize over the large experimental
    uncertainties and many nuisance parameters to infer the structural parameters
    of M31.
    We find that when we only consider so-called ``non-informative''
    parameterizations of our prior knowledge about the nuisance parameters, the
    data does not contain enough information to provide useful constraints on
    the physical galaxy parameters. Instead, by using physically motivated
    prior information about the stellar mass-to-light ratios, we find excellent
    constraints M31's structural parameters (TODO: what does this mean and how do
    we quantify it). TODO: describe the galaxy models here qualitative and
    quantitatively.

    To simultaneously marginalize and sample the posterior distribution, we
    we generalize and parallelize the affine invariant Markov chain Mont Carlo
    algorithm developed by Goodman \& Weare. This algorithm has only one tuning
    parameter --- instead of the $D\,(D+1)/2$ hyperparameters that must
    be fine-tuned using the standard methods --- making an expensive ``burn-in''
    phase unnecessary. This algorithm is especially useful in problems (like this
    one) where the likelihood evaluations require an expensive numerical
    computation.

\end{abstract}

\keywords{
methods: data analysis
---
methods: statistical
---
Local Group
---
galaxies: kinematics and dynamics
---
galaxies: structure
}

% -------- BODY --------

\defcitealias{ew}{EW00}
\defcitealias{vdm}{vdMG08}

\section{Introduction}

\begin{itemize}

\item Use of satellite galaxies as a probe of the Galactic potential
has had a long history dating back to Hartwick and Sargent (1978) (see
Little and Tremaine for an historical overview and further
references).  More recently, the methods have been applied to M31 (the
only other galaxy where accurate distances and velocities, relative to
the host galaxy, have been observed.  Early methods used simple mass
estimators.

\item Little and Tremaine introduced the method of Bayesian inference
to the problem for the Milky Way.  BI allows one to estimate
confidence intervals.  Furthermore, Bayesian inference allows one to
properly propagate errors in measurements (marginalization).  As well,
LT modelled the tracer population using a distribution function.

\item Further step was taken by Kochanek (1996) who used a more ``realistic''
and flexible distribution function (Jaffe model).

\item Evans and Wilkinson (1999) present the first such analysis of
M31.  Most of the previous estimates of M31 mass were based either on
the rotation curve, which extends to only 30 kpc.  Courteau and van
den Bergh (1999) used simple mass estimators for a sample of seven
satellites.

\item The basic idea of the Bayesian treatment is to build a
generative model for both the distribution function of the tracers and
the gravitational potential and then marginalize over the tracer DF.

\end{itemize}

\section{The Data}

\subsection{Satellite galaxy kinematics}

Table \ref{tab:satdata} lists the relevant information for the 23 satellite
galaxies in our sample.  This sample was constructed by supplementing the
catalogues from \citet{Evans:2000a} and \cite{McConnachie:2006} with more
recently discovered dwarf galaxies and updated distance and velocity measurements.
We restrict our sample to galaxies with measurements of both line-of-sight
velocity and distance.

\object[And IX]{Andromeda IX} and \object[And X]{X} were discovered by
\citet{Zucker:2004} and \citet{Zucker:2007} respectively in the Sloan Digital
Sky Survey \citep[\sdss;][]{York:2000} and subsequently observed spectroscopically
by \citet{Collins:2010} and \citet{Kalirai:2010}.  \citet{Kalirai:2010} also
refined the distance and line-of-sight velocity measurements for
\object[And I]{Andromeda I}, \object[And II]{II}, \object[And III]{III},
\object[And VII]{VII} and \object[And XIV]{XIV}.  \object[And XI]{Andromeda XI},
\object[And XII]{XII} and \object[And XIII]{XIII} were discovered by
\citet{Martin:2006} in a wide-field photometric survey of M31's outer halo
their distances and line-of-sight velocities were measured by
\citet{Collins:2010}.  \object[And XIV]{Andromeda XIV} was discovered by
\citet{Majewski:2007} and followed up by \citet{Kalirai:2010}.  The two most
recently discovered satellites with both line-of-sight velocity and distance
measurements -- \object[And XV]{Andromeda XV} and \object[And XVI]{XVI} -- were
both discovered by \citet{Ibata:2007} in the first year data from the
Pan-Andromeda Archaeological Survey (\pandas) and followed up by \citet{Letarte:2009}.

The distance measurement uncertainties for Andromeda IX, XI, XII and XIII are
extremely asymmetric about the mode and therefore present an interesting
inference problem.  In the limit of symmetric uncertainties, we can follow the
standard procedure of modeling the probability distribution function (PDF) of
the measurement as a Gaussian random variable.  This model breaks down, however,
when the uncertainties are significantly asymmetric.  Without samples from the PDF
of the original distance measurement, we are free to assume a parametric model
for the measurement PDF.  For simplicity, we choose a generalization of the
Gaussian distribution the skew-normal distribution (appendix \ref{sect:sn})
specified by one extra parameter.  Table \ref{tab:sn} lists the skew-normal
parameterizations of the asymmetric distance measurements.


\subsection{Rotation curve}

\label{sect:rcdata}

The rotation curve of M31 has been used extensively for mass modeling and it is
credited as one of the earliest arguments for dark matter \citep{Rubin:1970}.
Most measurements of gas kinematics are more than 20 years old and they are all
of quite low resolution.  \citet{Chemin:2009} recently published a high resolution
study of the HI kinematics in M31 (figure \ref{fig:rc_data}).  \Fig{rc_data}
shows the observed gas speed, position angle and inclination from the authors'
tilted-ring analysis.  We can see that the circular velocity profile in this
figure has significant kinematic structure not due to a smooth, axisymmetric
density distribution.  Comparing this to the plots of position angle and inclination
elucidates this even further.  In the inner $R_\mathrm{p} \lesssim 30^\prime \sim 7 \kpc$
and outer $R_\mathrm{p} \gtrsim 130^\prime \sim 30 \kpc$, the disk is much less
inclined than the median inclination and there is also significant scatter.
It is likely that these features are due to departures from axisymmetry and
non-circular motions (Cite Spekkens?).  \citet{Chemin:2009} argue that their
quoted uncertainties should account for some departure from axisymmetry but if
the velocity field is highly non-circular then it is unsuitable for dynamical
analysis at a na\"ive level.  With this in mind,

% Because of the uncertainty and scatter in the inner and outer regions of these plots, we only fit the data in the radial range $30^\prime \le R \le 130^\prime$.

(DFM: elaborate on the problems with this data)


\subsection{Surface brightness profile}

Larry: fill this section in...

% The kinematic data provides a strong constraint on the mass distribution of the
% galaxy but we should also recover the observed surface brightness profile for
% reasonable mass-to-light ratios of the disk and bulge.  For our observational
% constraint, we consider the classic measurement of the \Rband surface brightness
% profile from \citet{Walterbos:1987,Walterbos:1988}.  We choose to use the \Rband
% as opposed to a bluer band since it is less affected by dust and more sensitive
% to the low-mass stars that dominate the mass distribution
% \citep[e.g.][]{Binney:1998}.

\section{The Model}

To numerically generate the observations described in the previous section, we
use the model originally proposed by \citet{Kuijken:1995} and further improved
by \citet{Widrow:2008}.  The produced galaxy models consist of three axisymmetric,
dynamically distinct components --- a disk, a bulge and a dark matter halo.
By construction, they are simultaneously equilibrium solutions to the collisionless
Boltzmann equation and the self-gravitational Poisson equation \citep{Binney:2008}.
The phase-space distribution of particles in the model is fully specified by a
multi-component distribution function of three integrals of motion: the energy
\E, the angular momentum about the symmetry axis \Lz, and the vertical kinetic
energy of the disk stars \Ez
\begin{equation}
    \eqlabel{df1}
    \df (\E,\Lz,\Ez) = \dfhalo (\E) + \dfbulge (\E) + \dfdisk (\E,\Lz,\Ez).
\end{equation}

This distribution function is iteratively fine-tuned to match astrophysically
motivated density profiles for each component as described by \citet{Widrow:2008}.
The target densities that we choose are:
\begin{itemize}
    \item{an Einasto (CITATION?) halo,}
    \item{a de-projected S\'ersic profile \citep{Prugniel:1997,Terzic:2005}, and}
    \item{an exponential thin disk of mass \Md.}
\end{itemize}
It is important to note that although the target density profiles for the halo
and bulge are spherically symmetric, these components in the final model are
somewhat flattened by the disk potential \citep{Widrow:2008}.  In total, the
galaxy model is fully specified by nine physical parameters.  These and several
other input values are listed in \tab{params} alongside their assumed priors.




\section{The Method}

Since we would like to infer what basic structural parameters of M31 are consistent
with observation, we must Bayes' theorem to reconcile the model and the data.
In more precise terms, we wish to infer the probability, \posterior, of a set
of model parameters, \model, given the data, \data.  Conversely, what we can
easily construct is the likelihood of the data given a set of model parameters,
\likelihood.  Bayes' theorem allows us to relate these two quantities as
\begin{equation}
    \posterior \propto \prior \times \likelihood
\end{equation}
where \prior quantifies any prior knowledge we have about the model parameters.
To be more rigorous, all quantities in the above equation are also conditional
on our assumption that the model is applicable but we will not include this
explicitly for simplicity of notation.

In the subsequent sections, we outline the mathematical construction of \likelihood
and then discuss our choice of priors.

\subsection{Satellite galaxy kinematics}

\citet{Sales:2007b} analyzed the distribution of satellite galaxies formed in
$N$-body/gas-dynamical simulations including dark matter, gas and stars and found
that within the statistical errors, the satellites follow the same density
distribution as the dark matter.  Following this result, we assume that the
satellite galaxies are sampled directly from the dark matter distribution.
The true distribution of satellites is probably more complex than this simple
assumption \citep[e.g.][]{Ibata:2007}, however, given the sparseness of the
dataset, this is a necessary assumption.

Under this assumption, the phase space distribution function of satellite galaxies
is identically $\dfhalo (\E)$ (the halo DF from \eq{df1}) with
$\E = \Psi (s,z) + \left ( \vlos^2 + \vt^2 \right )/2$ where
$(s,z)$ are the M31-centric cylindrical coordinates of the satellite,
$\vlos \equiv v_\mathrm{S,A}^\mathrm{los}$ is given by \eq{vdm2}
and $v_\mathrm{t}$ is the unknown satellite velocity in the plane of the sky.

\subsubsection{Transformation to Andromeda-centric coordinates}

The positions and velocities of the satellite galaxies in our sample are all
measured in the heliocentric rest frame centered on the Sun but our model generates
points in a coordinate system centered on M31 in phase space. We must, therefore,
transform the observations into the rest frame of M31 in order to make any
comparisons to the model. The position space transformation has the form
\citep{Metz:2007}
\begin{equation}
    \bvec{r^\prime} = \bvec{R} (\bvec{r} - \bvec{r_\mathrm{M31}})
\end{equation}
where \bvec{r^\prime} and \bvec{r} are the position vectors of the satellite in
Andromeda-centric and heliocentric coordinates respectively, \bvec{r_\mathrm{M31}}
is the observed position of Andromeda and \bvec{R} is given by \eq{fullrotation}.
In appendix \ref{sect:coords}, we summarize the construction of this
transformation. Since the distance to M31 is a parameter in our model, we must
actively perform this transformation for every sample in our MCMC analysis.

\subsubsection{Velocity corrections}

The observed velocities must also be corrected for the three-dimensional velocity
of M31 itself. As we discussed previously, the transverse velocity of M31 is unknown
so we include it as a model parameter.

We begin by writing the full three-dimensional velocity of a given satellite
galaxy $S$ with respect to M31 (labeled $A$) as
\begin{equation}
    \eqlabel{vdm1}
    \vrel{S}{A} = \vrel{S}{\odot} + \vrel{\odot}{A}
\end{equation}
where $\mathbf{v}_{X,Y} = \mathbf{v}_{X}-\mathbf{v}_Y$ is the velocity of $X$
with respect to $Y$.  This equation can be rewritten
\begin{equation}
    \mathbf{v}_\mathrm{S,A} = \mathbf{v}_{\mathrm{S},\odot}
    + \mathbf{v}_{\odot,\mathrm{G}} + \mathbf{v}_{\mathrm{G},\odot}
    + \mathbf{v}_{\odot,\mathrm{A}} = \mathbf{v}_{\mathrm{S,G}}
    + \mathbf{v}_{\mathrm{G,A}}
\end{equation}
where $G$ indicates the Milky Way.  Then, we decompose $\mathbf{v}_{\mathrm{G,A}}$
into a component along our line-of-sight to M31 $\hat{l}_\mathrm{A}$ and a
component in the plane of the sky $\hat{l}_\mathrm{A}^\perp$
\begin{equation}
    \mathbf{v}_\mathrm{S,A} = \mathbf{v}_{\mathrm{S,G}} + (\mathbf{v}_{\mathrm{G,A}}
    \cdot \hat{l}_\mathrm{A}) \, \hat{l}_\mathrm{A} + (\mathbf{v}_{\mathrm{G,A}}
    \cdot \hat{l}_\mathrm{A}^\perp) \, \hat{l}_\mathrm{A}^\perp.
\end{equation}
$(\mathbf{v}_{\mathrm{G,A}} \cdot \hat{l}_\mathrm{A})$ has been tightly
constrained \citep{Courteau:1999} but
$\mathbf{v}_{\odot,\mathrm{A}} \cdot \hat{l}_\mathrm{A}^\perp$ is not directly
measurable. Therefore, all measurements of satellite line-of-sight velocities are
contaminated by the transverse velocity term.  \citet{ew} argue that
$\mathbf{v}_{\mathrm{G,A}} \cdot \hat{l}_\mathrm{A}^\perp$ should be negligible
since the Milky Way-M31 orbit should be nearly radial and the vector between the
Galactic centre and M31 is approximately equal to our line-of-sight
$\hat{l}_\mathrm{A}$.  This is not an unreasonable assumption since there are
no other large galaxies in the LG to generate significant angular momentum.
It is, however, an unnecessary assumption and in making it, the previous authors
are losing significant information about the proper motion.  Instead, we can
ctually exploit this contamination to obtain a probabilistic constraint on the
transverse motion of the M31 system.

\citet{Bahcall:1981} -- and subsequently \citet{ew} and \citet{vdm} -- pointed
out that the line-of-sight velocities of satellites at large angular distances
will be most significantly affected by this unmeasured systematic transverse
motion and dynamical modelling is highly sensitive to this.  \citet{vdm}
demonstrated, however, that this effect can be used to \emph{infer} the
systematic proper motion.  Specifically, if we project \eq{vdm1}
onto the line-of-sight of the satellite $\hat{l}_\mathrm{S}$ we find that the
peculiar velocity of a satellite along the line-of-sight is
\begin{eqnarray}
    \eqlabel{vdm2}
    v_\mathrm{S,A}^\mathrm{los} & =& \mathbf{v}_{\mathrm{S}}
    \cdot \hat{l}_\mathrm{S} - (\mathbf{v}_{\mathrm{A}}
    \cdot \hat{l}_\mathrm{A}) \, \cos \Phi_\mathrm{S} \nonumber \\ &&
    - (\mathbf{v}_{\mathrm{A}} \cdot \hat{l}_\mathrm{A}^\perp) \,
    \sin \Phi_\mathrm{S} \, \cos \left ( \Theta_\mathrm{S} - \Theta_\perp \right ).
\end{eqnarray}
where $\mathbf{v}_{\mathrm{S}}$ and $\mathbf{v}_{\mathrm{A}}$ are heliocentric
velocities and $\Phi_\mathrm{S}$ is the angular separation between
$\hat{l}_\mathrm{A}$ and $\hat{l}_\mathrm{S}$.  In \eq{vdm2},
$\Theta_\mathrm{S}$ and $\Theta_\perp$ are the position angles of the satellite
with respect to M31 and $\hat{l}_\mathrm{A}^\perp$ respectively measured from
north through east \citep[][their equation 3]{vdm}.

It is clear from \eq{vdm2} that the line-of-sight velocities of
the satellites will exhibit an oscillation as a function of position angle that
will depend on the transverse velocity of M31.  This effect will be strongest
for satellites at large angular separations.  This effect, though not immediately
visible in the dataset \citep[e.g.~figure 1 of][]{vdm}, can be used to constrain
the transverse velocity.  \citet{vdm} modelled $v_\mathrm{S,A}^\mathrm{los}$ as
a Gaussian velocity deviate with constant dispersion $\sigma$ along the line-of-sight
and inferred $\mathbf{v}_{\mathrm{A}} \cdot \hat{l}_\mathrm{A}^\perp$ by fitting
\eq{vdm2} to the data using least-squares (assuming uniform priors
in $\mathbf{v}_{\mathrm{A}} \cdot \hat{l}_\mathrm{A}^\perp$ and $\Theta_\perp$).

We choose to model $v_\mathrm{S,A}^\mathrm{los}$ with a realistic, cosmologically
motivated distribution function and include the transverse velocity of M31 as
a model parameter.  In doing so, we infer both the mass profile and the
transverse motion simultaneously and self-consistently.

To compare results with \citet{vdm}, we decompose
$\mathbf{v}_{\mathrm{A}} \cdot \hat{l}_\mathrm{A}^\perp$ into northern and
western components
\begin{eqnarray}
    & v_\mathrm{W} \equiv
    (\mathbf{v}_{\mathrm{A}} \cdot \hat{l}_\mathrm{A}^\perp)
        \, \cos \left ( \Theta_\perp + 90^\circ \right ) & \nonumber \\
    & v_\mathrm{N} \equiv
    (\mathbf{v}_{\mathrm{A}} \cdot \hat{l}_\mathrm{A}^\perp)
        \, \sin \left ( \Theta_\perp + 90^\circ \right ) & .
\end{eqnarray}
Repeating the fitting procedure from \citet{vdm} with our updated sample of
satellites, we find
$(v_\mathrm{M31,los}, v_\mathrm{W}, v_\mathrm{N}) = (-303, -52, 46) \pm (24, 197, 107)$
km s$^{-1}$ where
$v_\mathrm{M31,los} \equiv \mathbf{v}_{\mathrm{A}} \cdot \hat{l}_\mathrm{A}$.
This is in reasonably good agreement with the result from \citet{vdm} using only
satellite line-of-sight velocities:
$(v_\mathrm{M31,los}, v_\mathrm{W}, v_\mathrm{N}) = (-270, -136, 5) \pm (19, 148, 75)$
km s$^{-1}$.

\subsection{Computing the likelihood function}

\subsubsection{Satellite galaxy kinematics}

Based on the considerations of the previous section, the likelihood of observing
satellite $i$ at its measured position in phase space given a specific model $\Theta$ is
\begin{equation}
    \eqlabel{satprob1}
    \mathrm{p}_\mathrm{sat}(\mathbf{X}_i | \Theta,I) =
    \left . \int d^6 \mathbf{X} \, S(\mathbf{X} | \mathbf{X}_i) f_\mathrm{h} (E)
    \right /\int d^6 \mathbf{X} \, f_\mathrm{h} (E)
\end{equation}
where $S(\mathbf{X} | \mathbf{X}_i)$ is the distribution of the measured quantities
(including the uncertainties).  For our entire sample of satellite galaxies, three
components of position space (right ascension, declination and distance) and one
component of velocity space (line-of-sight) have been measured while the two
transverse velocity components are unknown.  Therefore, the first integral in
\eq{satprob1} can be rewritten
\begin{eqnarray}
    \eqlabel{satprob2}
    \int d^6 \mathbf{X} \, S(\mathbf{X} | \mathbf{X}_i) f_\mathrm{h} (E)
    & = & 2 \pi \, \int_{-\infty}^{\infty} dD \int_{-\infty}^{\infty} dv_\mathrm{obs}
    \int_0 ^{v_\mathrm{max}} v_{\mathrm{t}} \, dv_{\mathrm{t}}
    g(D,v_\mathrm{obs},v_\mathrm{t})
\end{eqnarray}
where
\begin{equation}
    g (D,v_\mathrm{obs},v_\mathrm{t})
    = \tilde{\mathcal{N}} (D | \xi_i, \omega_i, \alpha_i) \,
    \mathcal{N} (v_\mathrm{obs} | v_{\mathrm{obs},i},\delta v_{\mathrm{obs},i})\,
    f_\mathrm{h} (E)  ,
\end{equation}
$v_\mathrm{max} \equiv \sqrt{2 \Psi - v_\mathrm{los}^2}$ is the maximum transverse
component of the satellite's velocity;
$v_\mathrm{obs} \equiv \mathbf{v}_\mathrm{S} \cdot \hat{l}_\mathrm{S}$ is the
observed line-of-sight velocity;
$\mathcal{N} (v_\mathrm{los}; v_\mathrm{los,i},\delta v_\mathrm{los,i})$ is the
Gaussian PDF; and  $\tilde{\mathcal{N}} (D; \xi_i, \omega_i, \alpha_i)$ is the
skew-normal parameterization of the distance measurement.  The angular position
integrals are simply delta functions (because the measurement uncertainty is
negligible) so they have been omitted \eq{satprob2}.

We evaluate the integration in \eq{satprob2}  numerically using Monte
Carlo integration with importance sampling for $D$ and $v_\mathrm{obs}$ and
Simpson's rule for $v_{\mathrm{t}}$.  We calculate the full log-likelihood of the
satellite data given the model by taking the logarithm of \eq{satprob2}
and summing over the set of satellites
\begin{equation}
    \ln \mathrm{p}_\mathrm{s} (\mathbf{X}| \Theta, I)
    = \sum_i \ln \mathrm{p}_\mathrm{s} (\mathbf{X}_i | \Theta,I).
\end{equation}
Similarly, the normalization due to the second integral in \eq{satprob1} is
\begin{equation}
    \eqlabel{satnorm}
    N_\mathrm{s} \times
        \ln \left [ \int d^6 \mathbf{X} \, f_\mathrm{h} (E) \right ] =
        N_\mathrm{s} \ln M_\mathrm{h}
\end{equation}
where $N_\mathrm{s}$ is the number of satellites in the sample and the integral
is the total halo mass, by definition.  In practice, we only consider the mass
within some limiting radius $r_\mathrm{max}$ where $r_\mathrm{max}$ is approximately
the distance of the furthest satellite from M31 (the assumed edge of the survey).
We choose a constant $r_\mathrm{max} = 550$ kpc but the results are insensitive
to this choice.

\subsubsection{Rotation curve}

From the gravitational potential of our full galaxy model, we can calculate the
expected circular velocity $v_\mathrm{m}(r)$ at a given radius $r$ in the disk.
We compare this value to the observed value $v_i$ at radius $r_i$ assuming
uncorrelated Gaussian uncertainties
$\delta v_i = \sqrt{\delta v_{0,i}^2 + \epsilon_\mathrm{RC}}$ where
$\delta v_{0,i}^2$ is the published uncertainty and $\epsilon_\mathrm{RC}$ is a
free parameter --- called ``jitter'' --- that accounts for noise in the data and
allows the true uncertainties to be inferred.  For simplicity, we assume that
$\epsilon_\mathrm{RC}$ is a constant with radius.  The log-likelihood of the
observed values given a particular model specified by the set of parameters
$\mathbf{\theta}$ is
\begin{equation}
    \eqlabel{rclike}
    \ln \mathrm{p}_\mathrm{rc} (\{ r_i, v_i, \delta v_i \}| v_\mathrm{m},
    \epsilon_\mathrm{RC}) = - \sum_i \left [ \frac{[v_\mathrm{m}(r_i) - v_i]^2}
        {2 \delta v_i ^2} - \frac{1}{2} \ln (2 \pi \delta v_i ^2) \right ].
\end{equation}
This is simply the standard result that $p \propto e^{-\chi^2/2}$ but it is
important to recognize that the normalization coefficient is a function of
$\epsilon_\mathrm{RC}$ so it will not remain constant across models.
Intuitively, the second term on the righthand side of \eq{rclike} penalizes
models with large values of $\epsilon_\mathrm{RC}$ for a lack of specificity.

\subsubsection{Surface brightness profile}

As for the rotation curve, we assume Gaussian uncertainty and calculate the
likelihood of the observed surface brightness profile given our model.  In
particular, we assume that the uncertainty can be parameterized by a constant
variance $\epsilon_\mu$ at all radii.  We then calculate the surface brightness
profile of our model $\mu_R (r_i)$ in the $R$-band assuming constant
mass-to-light ratios for the disk and bulge and compare it to the data
$\mu_i$ (at radius $r_i$).  The likelihood of the data is therefore
\begin{equation}
    \eqlabel{sbpprob}
    \ln \mathrm{p}_\mathrm{sbp} (\{ r_i, \mu_i \} | \mu_R,\epsilon_\mu )
    = - \frac{1}{2} \sum_i \left [ \frac{\left [ \mu_R (r_i) - \mu_i \right ]^2 }
        {\epsilon_\mu} + \ln (2 \pi \epsilon_\mu) \right ]
\end{equation}
analogous to \eq{rclike}.



\subsection{Na\"ive (uninformative) priors}

\label{sect:priors}

\Tab{params} lists all the model parameters and their assumed priors.

For any dimensioned parameter that can vary from zero to infinity, it is best to
assume a logarithmic prior, especially if there is large uncertainty in the
physically allowed range \citep{Gregory:2005}.  The logarithmic prior enforces
scale invariance and it is implemented by sampling in the logarithm of the parameter.
To this end, we adopt uniform priors on the logarithm of the scale parameters in
our model unless a more informative prior is available.

For the $R$-band mass-to-light ratios of the disk and bulge, we set a conservative
hard upper limit of $\Upsilon_R \le 10 \, M_\odot /\mathcal{L}_{\odot,R}$ but
the data provides a much tighter constraint.  For the components of the systematic
transverse velocity, we assume a uniform prior on $v_\mathrm{W}$ and $v_\mathrm{N}$
but set a hard upper limit on the total transverse velocity of
$\sqrt{v_\mathrm{W}^2+v_\mathrm{N}^2} < 1000$ km s$^{-1}$ for computational purposes.

Throughout, we adopt a Gaussian prior on the distance with mean $785$ kpc and
standard deviation $25$ kpc \citep{McConnachie:2005,McConnachie:2006}.  Similarly,
for the systematic heliocentric line-of-sight velocity, we adopt the value
$-301 \pm 1$ km s$^{-1}$ \citep{Courteau:1999}.

\subsection{Sampling the posterior probability}

Combining the likelihood functions constructed in the previous section and
applying Bayes' theorem, we have that the posterior probability of the model
parameters given the full data set is
\begin{equation}
    \eqlabel{fullposterior}
    \ln \mathrm{p} (\mathbf{\Theta} | \mathbf{X}, I) =
    \ln \mathrm{p} (\mathbf{\Theta} | I) +  \ln \mathrm{p}_\mathrm{sat}
    + \ln \mathrm{p}_\mathrm{rc}
    + \ln \mathrm{p}_\mathrm{sbp} + \mathrm{constant}
\end{equation}

Our generative model of the data is numerical and none of the marginalization
integrals are analytic. Therefore, to estimate the value of \eq{fullposterior}
and perform the marginalization integrals over nuisance parameters, we must
rely on sampling. This is a large computational challenge since every evaluation
of \eq{fullposterior} requires substantial expensive computation. To reduce the
runtime, we use an extremely efficient sampling algorithm that employs an affine
invariant proposal. Intuitively, affine invariance means that the algorithm is
insensitive to the aspect ratio of the density even if it is highly anisotropic.
Traditional MCMC algorithms require an extensive ``burn-in''.

In practice, we are actually interested in the marginalized form of \eq{fullposterior}
\begin{equation}
    \eqlabel{marginalization}
    p (\tilde{\boldsymbol{\Theta}} | \mathbf{X}) \propto
        p (\tilde{\boldsymbol{\Theta}}) \int
        p (\mathbf{X} | \tilde{\boldsymbol{\Theta}},\boldsymbol{\alpha}) \,
        p (\boldsymbol{\alpha}) \, \dd \boldsymbol{\alpha}
\end{equation}
where $\tilde{\boldsymbol{\Theta}}$ is the vector of physical parameters of
interest. The integral in
\eq{marginalization} is over a set of \emph{nuisance parameters} $\boldsymbol{\alpha}$
that are (generally) of no physical interest in the problem. The function
$p (\mathbf{X} | \tilde{\boldsymbol{\Theta}},\boldsymbol{\alpha}) \, p (\boldsymbol{\alpha})$
is not integrable analytically. Instead, since the likelihood function is only
numerical, the integration in \eq{marginalization} is actually very expensive
to compute. In this regime, the numerical integration algorithm of choice is
generally Markov chain Monte Carlo (MCMC).
Since the likelihood function is expensive to calculate, it is advantageous to
use a sampling algorithm that reduces the necessary number of likelihood
evaluations. This also precludes the use of second order methods (such as
hybrid/Hamiltonian Monte Carlo) that require the calculation of (numerical)
gradients of the likelihood function.

Most uses of MCMC in the astrophysics literature are based on slight modifications
to the Metropolis-Hastings (M-H) method \citep[e.g.][]{MacKay:2003}. Each step in
a M-H chain is proposed using a multivariate Gaussian centered on the current
position of the chain. Since each term in the covariance matrix of this proposal
distribution is an unspecified parameter, this method has $D\,[D+1]/2$ tuning parameters
(where $D$ is the dimension of the parameter space).  To make matters worse, the
performance of this sampler is very sensitive to the optimality of these tuning
parameters and there is no fool-proof method for choosing the values correctly.
As a result, many heuristic methods have been developed to attempt to determine
the optimal parameters in a data-driven way \citep[e.g.][]{Gregory:2005,Dunkley:2005,Widrow:2008}.
Unfortunately, these methods all require ``burn-in'' phases where shorter Markov chains
are sampled and the results are used to tune the hyperparameters.

\subsection{MCMC Sampling algorithm}

\paragraph{The stretch move}

\citet{Goodman:2010} proposed an affine invariant ensemble sampling algorithm
informally called the ``stretch move''. This method involves simultaneously
evolving an ensemble of $K$ \emph{walkers} where the proposal distribution for one
walker $k$ is based on the current positions of the $K-1$ walkers in the
\emph{complementary ensemble}.

In practice, to update the position of a walker at position $\mathbf{X}_k$,
another walker $\mathbf{X}_j$ with $j \ne k$ is randomly chosen and then
a new position is proposed:
\begin{equation}
    \eqlabel{proposal}
    \mathbf{X}_k (t) \rightarrow \mathbf{Y} = \mathbf{X}_j + Z \, [\mathbf{X}_k (t) - \mathbf{X}_j]
\end{equation}
where $Z$ is a random variable drawn from a distribution $g(z)$.  It is clear that
if $g(z)$ satisfies
\begin{equation}
    g(z^{-1}) = z \, g(z),
\end{equation}
the proposal of \eq{proposal} is symmetric. In this case, the chain will satisfy
detailed balance if the proposal is accepted with probability
\begin{equation}
    \min \left \{ 1, Z^{n-1} \, \frac{p(\mathbf{Y})}{p(\mathbf{X}_k(t))} \right \}.
\end{equation}
This procedure is then repeated for each walker in the ensemble \emph{in series}.

\citet{Goodman:2010} advocate for a particular form of $g(z)$, namely
\begin{equation}
    g(z) \propto \left \{ \begin{array}{ll}
        \displaystyle\frac{1}{\sqrt{z}} & \mathrm{if}\, z\in \left [ \displaystyle\frac{1}{a}, a \right ], \\
        0 & \mathrm{otherwise}
    \end{array} \right .
\end{equation}
where $a$ is an adjustable scale parameter. For comparison, I will also use this
distribution with $a=2$ for all benchmarks in this paper.

\paragraph{The parallel stretch move}

It is tempting to na\"ively parallelize the stretch move algorithm by simultaneously
advancing each walker based on the state of the ensemble instead of evolving the
walkers in series. Unfortunately, this would no longer satisfy detailed balance.
Instead, if we split the ensemble into two ensembles (labeled \emph{blue} and
\emph{red} for convenience) and simultaneously update all the blue walkers ---
using the stretch move procedure --- based on the positions of \emph{only the red
walkers} then the outcome is a valid step for each of the walkers. Then,
the red walkers are advanced based only on the positions in the blue ensemble.

The performance of this method --- quantified by the autocorrelation time --- is
comparable to the traditional stretch move algorithm but the fact that one can now
take advantage of generic parallelization makes this generalization extremely
powerful.




% To numerically marginalize over the nuisance parameters and approximate posterior
% probability distribution function, we use the Markov chain Monte Carlo method
% developed by \citet{Goodman:2010} and first applied to a problem in astrophysics
% by \citet{Hou:2011}.  This MCMC sampler simultaneously evolves an affine-invariant
% ensemble of ``walkers'' with the Metropolis acceptance probability.  Each walker
% is essentially a standard Metropolis-Hastings chain but the proposal distribution
% depends on the distribution of the other members of the ensemble.  While this means
% that the evolution of a single walker is not Markov, \citet{Goodman:2010} show that
% the results of the ensemble are.  This method is generally more powerful than the
% standard MCMC techniques employed in the field \citep[e.g.~][]{Dunkley:2005}
% because it is equally efficient at sampling \emph{any} convex surface of a given
% dimensionality, regardless of its shape \citep{Goodman:2010}.  Most other samplers
% require a highly fine-tuned proposal distribution whose optimal value is not known
% a priori \citep[e.g.][]{MacKay:2003,Gregory:2005}.  The standard method of
% empirically determining the optimal proposal distribution involves an extended
% ``burn-in'' period where a ``heated'' distribution is sampled and therefore, the
% samples from this period must be discarded.

% The \citet{Goodman:2010} sampler is composed of $n$-walkers with positions
% $\{ \mathbf{X}_1, \mathbf{X}_2, \ldots, \mathbf{X}_n \}$ in the $M$-dimensional
% parameter space of the distribution.  The positions of the walkers are then updated
% in series.  To update walker $\mathbf{X}_j (t) \rightarrow \mathbf{X}_j (t+1)$,
% another walker $\mathbf{X}_k$ is chosen at random from the conjugate set
% $\{ \mathbf{X}_1 (t+1), \ldots, \mathbf{X}_{j-1} (t+1), \mathbf{X}_{j+1} (t),
%     \ldots, \mathbf{X}_{n} (t)  \}$.  Then, the proposal position
% \begin{equation}
%     \eqlabel{strech}
%     \mathbf{\tilde{X}}_j(t+1) = \mathbf{X}_k + Z(\mathbf{X}_j(t)-\mathbf{X}_k).
% \end{equation}
% is accepted with the probability
% \begin{equation}
%     \mathrm{min} \left ( 1, Z^{M-1}
%         \frac{P\left [ \mathbf{\tilde{X}}_j(t+1)\right ]}
%         {P\left [ \mathbf{X}_j(t)\right ]} \right ).
% \end{equation}
% Following \citet{Goodman:2010}, we sample $Z$ from
% \begin{equation}
%     \left \{ \begin{array}{cl}
%         \displaystyle{\frac{1}{\sqrt{Z}}} & \displaystyle{\frac{1}{a} < Z < a}, \\
%         0 & \mathrm{otherwise}\\
%     \end{array}  \right .
% \end{equation}
% with $a>1$ so that \eq{strech} satisfies detailed balance.  In all that follows,
% we employ an ensemble of 100 walkers and a constant $a = 2$.

\section{Results}

DFM: Introduce here

- MCMC stats

- Summarize and introduce data subsets

- refer to section \ref{sect:rcdata} about rotation curve data problems.


\subsection{Connection With Cosmology}

DFM: outline virial parameters/calculation



\subsection{Constraints From Individual Datasets}

\subsubsection{Satellite kinematics}

\subsubsection{Rotation curve}

\subsection{Full Analysis}

DFM:

- Full and clipped rotation curve

- First: summarize, etc.

\subsubsection{Convergence to observational constraints}

\paragraph{Rotation curve}

\paragraph{Surface brightness profile}



\subsubsection{Galaxy model parameters}

\subsubsection{Mass profile}

\subsubsection{Problems}

\paragraph{Mass-to-light ratios}

\paragraph{Halo concentration}

\paragraph{Mass at large radii}


\subsection{Stability Considerations}

% \citet{Widrow:2008} presented a numerical the stability of a galactic disk is very sensitive to the relative disk mass.

Near-infrared imaging and kinematic arguments suggest that M31 has a (weak?) bar \citep{Beaton:2007,Saglia:2010}, presenting a strong departure from axisymmetry in the central regions of the galaxy.  Many attempts have been made to confront this complication when modeling the Galaxy by constructing and numerically evolving axisymmetric N-body models and then comparing to observations \citet{Ostriker:1973,Sellwood:1985,Fux:1997,Widrow:2008}.  The most important conclusion for our purposes is that the stability of a galactic disk is very sensitive to the relative disk mass.  It has been shown \citep{Debattista:2000,Widrow:2008} that the ratio of total circular velocity $v_\mathrm{tot}$ and the contribution of an isolated disk $v_\mathrm{d}$ measured at 2.2 disk scale lengths
\begin{equation}
    X^\prime = \frac{v_\mathrm{tot}^2 (2.2\,R_\mathrm{d})}{v_\mathrm{d}^2 (2.2\,R_\mathrm{d})}
\end{equation}
is a good proxy for estimating the stability of a disk.  In particular, \citet{Widrow:2008} showed that this parameter is equivalent to the Goldreich-Tremaine $X$ stability parameter \citep{Goldreich:1978,Goldreich:1979} with negligible scatter.  The disk density profile of a galaxy model with $X^\prime \lesssim 3$ (Larry:  is this number right?) becomes extremely distorted when evolved numerically\footnote{See http://www.cita.utoronto.ca/~dubinski/DynamicalBlueprints/ for a visualization.} and the resulting disk is qualitatively very different than observed.

Although the parameter $X^\prime$ doesn't provide a rigorous, quantitative constraint on our models, we can consider a grid of lower limits on $X^\prime$ to determine its effect on our inference.  We impose a cubic prior on $X^\prime$ varying smoothly from zero at some $X_\mathrm{min}$ to unity at $X_\mathrm{min}+\Delta X$.

DFM: Right now I've only done $(X_\mathrm{min},\Delta X) = (3,1)$ should also do: $X_\mathrm{min} = 2,2.5,3.5,4$? or something like that.




\subsection{Proposed Model}

DFM: Argue for a specific model here ?



\section{Conclusions}




% -------- ACKNOWLEDGEMENTS --------

\acknowledgments The authors would like to thank T.J.~Bridges, S.~Courteau, D.A.~Hanes and
D.W.~Hogg for the insightful conversations and all their support.

This research made extensive use of the Python programming language and the
open-source Python packages SciPy, NumPy, matplotlib and MarkovPy.  It also
made us of A. Collette's HDF5 for Python (http://h5py.alfven.org/). All code
and data used in this project are available from the authors upon request.

% -------- REFERENCES --------

\bibliographystyle{apj}
\bibliography{/Users/dfm/Documents/BibDesk/database.bib}


% -------- APPENDICES --------

\appendix

\section{The Skew-Normal Distribution}
\label{sect:sn}


The skew-normal density function is given by \citep{OHagan:1976a,Azzalini:1996}
\begin{equation}
    \eqlabel{sn}
    \tilde{\mathcal{N}} (x | \xi, \omega, \beta) = \left ( \frac{2}{\omega} \right ) \phi \left ( \frac{x-\xi}{\omega} \right ) \, \Phi \left [ \beta \left ( \frac{x-\xi}{\omega} \right ) \right ],
\end{equation}
where
\begin{equation}
    \phi (x) = \frac{1}{\sqrt{2\pi}} e^{-x^2/2}
\end{equation}
is the standard normal distribution and
\begin{equation}
    \Phi(x) = \int _{-\infty} ^x \phi(x^\prime) \, dx^\prime = \frac{1}{2} \left [ 1 + \mathrm{erf} \left ( \frac{x}{\sqrt{2}} \right )\right ]
\end{equation}
is the cumulative distribution of $\phi(x)$.  In the limit $\beta \rightarrow 0$ we recover the standard normal distribution with mean $\xi$ and width $\omega$.  The skew-normal distribution can also be efficiently sampled pseudo-randomly.



\section{Change of Coordinate Basis}

\label{sect:coords}

To convert from observed right ascension $\alpha$, declination $\delta$ and distance $r$ to M31-centric coordinates, we start with a position $\mathbf{r}$ measured in the equatorial basis
\begin{equation}
    S \equiv \left ( \begin{array}{c}
        \mathbf{e}_x \\ \mathbf{e}_y \\ \mathbf{e}_z
    \end{array} \right ) = \left ( \begin{array}{c}
        \cos \delta \cos \alpha \\
        \cos \delta \sin \delta \\
        \sin \delta
    \end{array} \right )
\end{equation}
centered on the Sun.  We then transform to a basis $S^\prime \equiv (\mathbf{e}_x^\prime, \mathbf{e}_y^\prime,\mathbf{e}_z^\prime)^\mathrm{T}$ centered on M31 with equatorial coordinates $\mathbf{r}_\mathrm{M31}$ and basis vectors $\mathbf{e}_x^\prime$ and $\mathbf{e}_y^\prime$ in the plane of M31's disk and $\mathbf{e}_z^\prime$ normal to this plane.  $\mathbf{e}_x^\prime$ is the projection of our line-of-sight onto the disk.  The full rotation corrects for (a) the orientation of the ``normal triad'' at the position of M31, (b) the position angle of the stellar disk and (c) the inclination of the disk with respect to the line-of-sight.

The final transformation from $\mathbf{r} \in S$ to $\mathbf{r}^\prime \in S^\prime$ takes the form
\begin{equation}
    \mathbf{r}^\prime = \mathbf{R}(\mathbf{r} - \mathbf{r}_\mathrm{M31})
\end{equation}
where $\mathbf{R}$ is a rotation matrix that corrects for the effects mentioned previously.  To correct for the orientation of the triad at M31's position, we multiply $S$ by $\mathbf{R}_{rpq} (\alpha_\mathrm{M31}, \delta_\mathrm{M31})$ where
\begin{equation}
    \eqlabel{triad}
    \mathbf{R}_{rpq} (\alpha, \delta) \equiv \left ( \begin{array}{ccc}
    \cos \delta \cos \alpha & \cos \delta \sin \alpha & \sin \delta \\
    -\sin \alpha & \cos \alpha & 0 \\
    -\sin \delta \cos \alpha & -\sin \delta \sin \alpha & \cos \delta
    \end{array} \right ).
\end{equation}
With this transformation and the corrections for inclination and position angle, the full rotation becomes
\begin{equation}
    \eqlabel{fullrot}
    \mathbf{R} = \mathbf{R}_y (90^\circ - i) \, \mathbf{R}_x (\theta - 90^\circ) \, \mathbf{R}_{rpq} (\alpha_\mathrm{M31}, \delta_\mathrm{M31})
\end{equation}
where $\mathbf{R}_i$ ($i\in \{x,y,z\}$) is the \emph{right-handed} Cartesian rotation matrix about the given axis.  In \eq{fullrot}, $i = 77.5^\circ$ is M31's inclination and $\theta = 37.7^\circ$ is the position angle of the stellar disk measured from north over east \citep{de-Vaucouleurs:1958}.  Numerically, this matrix is the constant
\begin{equation}
    \eqlabel{fullrotation}
    \mathbf{R} \equiv \left ( \begin{array}{ccc}
0.7752 & 0.3204 & 0.5445 \\
-0.6261 & 0.5042 & 0.5947 \\
-0.0840 & -0.8019 & 0.5915
\end{array} \right )
\end{equation}
which differs from the transformation from \citet{Metz:2007} by a $z$-rotation of $\sim 0.4^\circ$ since they define $\mathbf{e}_z^\prime$ in the direction of the Galactic centre instead of the Sun.  For our axisymmetric purposes, this rotation is unnecessary.



% -------- MEDIA --------

\clearpage

% \begin{figure}[tbp]
% \epsscale{1.}
% \plotone{images/rc_data.eps}
% \caption{The observation of rotation speed (top), position angle (middle) and inclination (bottom) from \citet{Chemin:2009} with their quoted uncertainties.  The dashed lines in the bottom two plots indicate the median of the measurements and the dotted line shows the value adopted by \citet{Chemin:2009}.  The shaded region is the radial range that we consider in this work. \label{fig:rc_data}}
% \end{figure}

% % \begin{figure}[tbp]
% % \epsscale{1.}
% % \plotone{images/sbp_data.eps}
% % \caption{$R$-band surface brightness profile along the major axis of M31 from \citet{Walterbos:1987}.  The solid line is an illustrative disk/bulge decomposition for constant mass-to-light ratios.  The filled points are included in our fit and the open points are not. \label{fig:sbp}}
% % \end{figure}

% \begin{figure}[tbp]
% \epsscale{1.}
% \plottwo{images/bulge_disp.eps}{images/bulge_osc.eps}
% \caption{\emph{Top}: The projected line-of-sight velocity dispersion profile from \citet{Saglia:2010} plotted for all position angles.  \emph{Bottom}: The same data from the shaded region ($50^{\prime\prime} \le R \le 70^{\prime\prime}$).  The black line is the maximum likelihood estimate of the oscillatory behaviour with a mean $172 \pm 5$ km s$^{-1}$.\label{fig:bulge_disp}}
% \end{figure}



% results


% -------- TABLES --------

\clearpage

\begin{deluxetable}{l r rrrr r}
    \tablecaption{Our sample of satellite galaxies. \label{tab:satdata}}
    \tablewidth{0pt}
    \tablehead{
\colhead{Galaxy }   & \colhead{R.A. } & \colhead{Dec.} & \colhead{$D$ } & \colhead{$v_\mathrm{los}$ } & \colhead{Sources} \\
    & \colhead{(J2000.0)    } & \colhead{(J2000.0)}   & \colhead{(kpc)} & \colhead{(km s$^{-1}$) } &
    }
    \startdata

          M33       &   $1^h 33^m 50.9^s$   &   $+30^\circ 39^\prime 36.8^{\prime\prime}$   &   $809 \pm 24$        &   $-180 \pm 1$        &   1,2 \\
          M32       &   $0^h 42^m 41.8^s$   &   $+40^\circ 51^\prime 54.6^{\prime\prime}$   &   $785 \pm 25$        &   $-205 \pm 3$        &   1,2 \\
         IC 10      &   $0^h 20^m 17.3^s$   &   $+59^\circ 18^\prime 13.6^{\prime\prime}$   &   $660 \pm 65$        &   $-344 \pm 5$        &   1 \\
        NGC 205     &   $0^h 40^m 22.1^s$   &   $+41^\circ 41^\prime 7.1^{\prime\prime}$    &   $824 \pm 27$        &   $-244 \pm 3$        &   1,2 \\
        NGC 185     &   $0^h 38^m 58.0^s$   &   $+48^\circ 20^\prime 14.6^{\prime\prime}$   &   $616 \pm 26$        &   $-202 \pm 7$        &   1,2 \\
        IC 1613     &   $1^h 4^m 47.8^s$    &   $+2^\circ 7^\prime 4.0^{\prime\prime}$      &   $715 \pm 35$        &   $-232 \pm 5$        &   1 \\
        NGC 147     &   $0^h 33^m 12.1^s$   &   $+48^\circ 30^\prime 31.5^{\prime\prime}$   &   $675 \pm 27$        &   $-193 \pm 3$        &   1,2 \\
        Pegasus     &   $23^h 28^m 36.2^s$  &   $+14^\circ 44^\prime 34.5^{\prime\prime}$   &   $919 \pm 30$        &   $-182 \pm 2$        &   1,2 \\
         LGS 3      &   $1^h 3^m 55.0^s$    &   $+21^\circ 53^\prime 6.0^{\prime\prime}$    &   $769 \pm 23$        &   $-286 \pm 4$        &   1,2 \\
         And I      &   $0^h 45^m 39.8^s$   &   $+38^\circ 2^\prime 28.0^{\prime\prime}$    &   $745 \pm 24$        &   $-375.8 \pm 1.4$    &   1,2,3 \\
        And II      &   $1^h 16^m 29.8^s$   &   $+33^\circ 25^\prime 9.0^{\prime\prime}$    &   $652 \pm 18$        &   $-193.6 \pm 1.0$    &   1,2,3 \\
        And III     &   $0^h 35^m 33.8^s$   &   $+36^\circ 29^\prime 52.0^{\prime\prime}$   &   $749 \pm 24$        &   $-345.6 \pm 1.8$    &   1,2,3 \\
         And V      &   $1^h 10^m 17.1^s$   &   $+47^\circ 37^\prime 41.0^{\prime\prime}$   &   $774 \pm 28$        &   $-403 \pm 4$        &   1,2 \\
        And VI      &   $23^h 51^m 46.3^s$  &   $+24^\circ 34^\prime 57.0^{\prime\prime}$   &   $783 \pm 25$        &   $-354 \pm 3$        &   1,2 \\
        And VII     &   $23^h 26^m 31.0^s$  &   $+50^\circ 41^\prime 31.0^{\prime\prime}$   &   $763 \pm 35$        &   $-309.4 \pm 2.3$    &   2,3 \\
        And IX      &   $0^h 52^m 51.1^s$   &   $+43^\circ 11^\prime 48.6^{\prime\prime}$   &   $765^{+5}_{-150}$   &   $-207.7 \pm 2.7$    &   4,5 \\
         And X      &   $1^h 6^m 33.7^s$    &   $+44^\circ 48^\prime 15.8^{\prime\prime}$   &   $703 \pm 70$        &   $-163.8 \pm 1.2$    &   3,6 \\
        And XI      &   $0^h 46^m 21.0^s$   &   $+33^\circ 48^\prime 22.0^{\prime\prime}$   &   $760^{+10}_{-150}$  &   $-419.6 \pm 4.4$    &   5,7 \\
        And XII     &   $0^h 47^m 27.0^s$   &   $+34^\circ 22^\prime 29.0^{\prime\prime}$   &   $830^{+170}_{-30}$  &   $-558.4 \pm 3.2$    &   5,7 \\
        And XIII    &   $0^h 51^m 51.0^s$   &   $+33^\circ 0^\prime 16.0^{\prime\prime}$    &   $910^{+30}_{-160}$  &   $-195.0 \pm 8.4$    &   5,7 \\
        And XIV     &   $0^h 51^m 35.0^s$   &   $+29^\circ 41^\prime 49.0^{\prime\prime}$   &   $740 \pm 74$        &   $-481.0 \pm 2.0$    &   3,8 \\
        And XV      &   $1^h 14^m 18.7^s$   &   $+38^\circ 7^\prime 3.0^{\prime\prime}$     &   $630 \pm 60$        &   $-339 \pm 7$        &   9,10 \\
        And XVI     &   $0^h 59^m 29.8^s$   &   $+32^\circ 22^\prime 36.0^{\prime\prime}$   &   $525 \pm 50$        &   $-385 \pm 6$        &   9,10 \\


        \enddata
        \tablecomments{Sources: 1 - \citet{Evans:2000a}, 2 - \citet{McConnachie:2006}, 3 - \citet{Kalirai:2010}, 4 - \citet{Zucker:2004}, 5 - \citet{Collins:2010}, 6 - \citet{Zucker:2007}, 7 - \citet{Martin:2006}, 8 - \citet{Majewski:2007}, 9 - \citet{Ibata:2007}, 10 - \citet{Letarte:2009}}

\end{deluxetable}


\begin{deluxetable}{cccc}
    \tablecaption{Skew-normal parameterization of distance measurements \label{tab:sn}}
    \tablewidth{0pt}
    \tablehead{
    \colhead{Galaxy} & \colhead{Location ($\xi$)} & \colhead{Scale ($\omega$)} & \colhead{Shape ($\beta$)}
    }
    \startdata

        And IX      &   730     &   115     &   -7  \\
        And XI      &   717     &   107     &   -4  \\
        And XII     &   889     &   109     &   2   \\
        And XIII    &   854     &   103     &   -2  \\

    \enddata
\end{deluxetable}


\begin{deluxetable}{lll}
    \tabletypesize{\footnotesize}
    \tablecaption{Model parameters and their prior distributions \label{tab:params}}
    \tablewidth{0pt}
    \tablehead{\colhead{Parameter} & \colhead{Description} & \colhead{Prior}}
    \startdata


    $\alpha_\mathrm{M31}$                                   &   Right ascension (J2000.0)\tablenotemark{a}  & $00^\mathrm{h} 42^\mathrm{m} 44.4^\mathrm{s}$\\
    $\delta_\mathrm{M31}$                                   &   Declination (J2000.0)\tablenotemark{a}      & $+41^\circ 16^\prime 08^{\prime\prime}$\\
    $\theta$                                                &   Position Angle\tablenotemark{a}             & $77.5^\circ$ \\
    $i$                                                     &   Inclination\tablenotemark{a}                & $37.7^\circ$ \\

    \tableline

    $D_\mathrm{M31}/\mathrm{kpc}$                           &   Distance to M31\tablenotemark{b}            & $\mathcal{N} (785, 25^2)$ \\
    $v_\mathrm{M31,los}/\mathrm{km\,s}^{-1}$                &   Line-of-sight velocity\tablenotemark{c}     & $\mathcal{N} (301, 1)$ \\
    $(v_\mathrm{W},v_\mathrm{N})/\mathrm{km\,s}^{-1}$       &   Systemic transverse velocity                & $\sqrt{v_\mathrm{W}^2+v_\mathrm{N}^2} < 1000$ \\

    \tableline

    $\log_{10} \, a_\mathrm{h}/\mathrm{kpc}$                &   Halo scale length                           & $U(0,3)$  \\
    $\log_{10} \, v_\mathrm{h}/\mathrm{km\,s}^{-1}$         &   Halo characteristic velocity                & $U(0,4)$ \\
    $\log_{10} \, M_\mathrm{d}/10^{9} \, M_\odot$           &   Disk mass scale                             & $U(0.4,2.4)$  \\
    $\log_{10} \, R_\mathrm{d}/\mathrm{kpc}$                &   Disk scale length                           & $U(0,1)$  \\
    $\Upsilon_\mathrm{d}$                                   &   Disk $R$-band mass-to-light ratio           & $U(0,10)$  \\

    $\log_{10} \, v_\mathrm{b}/\mathrm{km\,s}^{-1}$         &   Bulge characteristic velocity               & $U(1,3)$ \\
    $\log_{10} \, R_\mathrm{e}/\mathrm{kpc}$                &   Bulge scale length                          & $U(-2,0.9)$  \\
    $n$                                                     &   Bulge S\'ersic index                        & $2$  \\
    $\Upsilon_\mathrm{b}$                                   &   Bulge $R$-band mass-to-light ratio          & $U(0,10)$  \\

    \tableline

    $\log_{10}\,\sqrt{\epsilon_\mathrm{RC}}/\mathrm{km\,s}^{-1}$   &   Rotation curve noise parameter              & $U(-1,6)$  \\
    $\log_{10}\,\sqrt{\epsilon_\mathrm{b}}/\mathrm{km\,s}^{-1}$    &   Bulge velocity dispersion noise parameter   & $U(-12,12)$  \\
    $\log_{10}\,\sqrt{\epsilon_\mu}/\mathrm{mag\,arcsec}^{-2}$                         &   Surface brightness profile noise parameter  & $U(-8,6)$  \\


    \enddata

    \tablenotetext{a}{\citet{de-Vaucouleurs:1958,de-Vaucouleurs:1991}}
    \tablenotetext{b}{\citet{McConnachie:2005}}
    \tablenotetext{c}{\citet{Courteau:1999}}
\end{deluxetable}




\begin{deluxetable}{lccc}
    % \tabletypesize{\footnotesize}
    \tablecaption{Model parameters and their prior distributions \label{tab:results}}
    \tablewidth{0pt}
    \tablehead{\colhead{Parameter} & \colhead{All} & \colhead{No fast} & \colhead{Stability}}
    \startdata


    $D_\mathrm{M31}/\mathrm{kpc}$                                   &    $755^{+16}_{-18}$       & $766^{+17}_{-18}$       & $765 \pm 16$           \\
    $v_\mathrm{W}/\mathrm{km\,s}^{-1}$                              &    $28 \pm 248$            & $12 \pm 214$            & $27 \pm 151$           \\
    $v_\mathrm{N}/\mathrm{km\,s}^{-1}$                              &    $58 \pm 126$            & $-15 \pm 116$           & $-7 \pm 76$            \\
    $\rho_\mathrm{WN}$                                              &    0.238                   & 0.101                   & 0.222                  \\\tableline
    $\log_{10} \, a_\mathrm{h}/\mathrm{kpc}$                        &    $1.80^{+0.14}_{-0.12}$  & $1.73^{+0.15}_{-0.18}$  & $0.91^{+0.26}_{-0.11}$ \\
    $\log_{10} \, v_\mathrm{h}/\mathrm{km\,s}^{-1}$                 &    $2.63 \pm 0.03$         & $2.62^{+0.04}_{-0.03}$  & $2.68^{+0.03}_{-0.02}$ \\
    $\log_{10} \, M_\mathrm{d}/10^{9} \, M_\odot$                   &    $2.16^{+0.04}_{-0.03}$  & $2.15^{+0.05}_{-0.03}$  & $1.81^{+0.03}_{-0.05}$ \\
    $\log_{10} \, R_\mathrm{d}/\mathrm{kpc}$                        &    $0.69^{+0.04}_{-0.03}$  & $0.68^{+0.05}_{-0.02}$  & $0.74 \pm 0.01$        \\
    $\Upsilon_\mathrm{d}$                                           &    $5.42^{+1.31}_{-1.26}$  & $4.53^{+1.77}_{-0.54}$  & $2.80^{+0.22}_{-0.29}$ \\
    $\log_{10} \, v_\mathrm{b}/\mathrm{km\,s}^{-1}$                 &    $2.28^{+0.23}_{-0.63}$  & $2.16^{+0.29}_{-0.67}$  & $2.66^{+0.06}_{-0.09}$ \\
    $\log_{10} \, R_\mathrm{e}/\mathrm{kpc}$                        &    $0.00^{+0.21}_{-0.30}$  & $-0.05^{+0.17}_{-0.47}$ & $0.07^{+0.06}_{-0.04}$ \\
    $\Upsilon_\mathrm{b}$                                           &    $0.90^{+5.88}_{-0.47}$  & $1.17^{+6.62}_{-0.77}$  & $1.77^{+0.74}_{-0.59}$ \\\tableline
    $\log_{10}\,\sqrt{\epsilon_\mathrm{RC}}/\mathrm{km\,s}^{-1}$    &    $1.39 \pm 0.03$         & $1.39 \pm 0.03$         & $1.49^{+0.08}_{-0.04}$ \\
    $\log_{10}\,\sqrt{\epsilon_\mathrm{b}}/\mathrm{km\,s}^{-1}$     &    $2.19^{+0.55}_{-0.37}$  & $2.16^{+0.43}_{-0.33}$  & $1.33^{+0.77}_{-1.22}$ \\
    $\log_{10}\,\sqrt{\epsilon_\mu}/\mathrm{mag\,arcsec}^{-2}$      &    $-0.70^{+0.22}_{-0.60}$ & $-0.59^{+0.10}_{-0.43}$ & $-1.24^{+0.26}_{-0.10}$\\


    \enddata
\end{deluxetable}


\end{document}
