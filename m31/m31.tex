% Work in progress
%   (c) Daniel Foreman-Mackey, all rights reserved


% TODO
% ----
% - Be clear about which band the M/Ls are in throughout
% - Name the two runs throughout:
%   * naive: run A
%   * m2l: run B
% - Re-arrange to make the run including the M/L constraint be the main run
%   and then just discuss the naive run in the discussion section
% - make it clear what parameterization of the density we're using.


\documentclass[preprint]{aastex}

\usepackage{amssymb,amsmath}

\usepackage{color,hyperref}
\definecolor{linkcolor}{rgb}{0,0,0.5}
\hypersetup{
    colorlinks=true,
    linkcolor =linkcolor,
    citecolor =linkcolor,
    filecolor =linkcolor,
    urlcolor  =linkcolor
}

\usepackage{multirow}

%% My definitions
\newcommand{\foreign}[1]{\emph{#1}}
\newcommand{\etal}{\foreign{et\,al.}}
\newcommand{\eg}{\foreign{e.g.}}
\newcommand{\etc}{\foreign{etc.}}

\newcommand{\project}[1]{\emph{#1}}
\newcommand{\sdss}{\project{SDSS}}
\newcommand{\pandas}{\project{PAndAS}}

\newcommand{\andromeda}{\object[M31]{Andromeda}}

% Convenience functions for referencing equations, figures, etc.
% NOTE: the reference key is automatically prefixed with fig:, eq:, etc.
\newcommand{\figlabel}[1]{\label{fig:#1}}
\newcommand{\Fig}[1]{Figure \ref{fig:#1}}
\newcommand{\fig}[1]{Figure \ref{fig:#1}}
\newcommand{\tablabel}[1]{\label{tab:#1}}
\newcommand{\Tab}[1]{Table \ref{tab:#1}}
\newcommand{\tab}[1]{Table \ref{tab:#1}}
\newcommand{\eqlabel}[1]{\label{eq:#1}}
\newcommand{\Eq}[1]{Equation (\ref{eq:#1})}
\newcommand{\eq}[1]{Equation (\ref{eq:#1})}

% math symbols
\newcommand{\dd}{\mathrm{d}}
\newcommand{\bvec}[1]{\ensuremath{\boldsymbol{#1}}}
\newcommand{\paramvector}[1]{\bvec{#1}}
\newcommand{\normal}[3]{\ensuremath{\mathcal{N} (#1 | #2, #3)}}
\renewcommand{\vector}[1]{\ensuremath{\bvec{#1}}}
\renewcommand{\matrix}[1]{\ensuremath{\bvec{#1}}}

% Bayesianism
\newcommand{\model}{\vector{\Theta}}
\newcommand{\data}{\vector{X}}
\newcommand{\prob}{\ensuremath{p}}
\newcommand{\probability}[2]{\ensuremath{\prob ( #1 | #2 )}}
\newcommand{\likelihood}{\probability{\data}{\model}}
\newcommand{\posterior}{\probability{\model}{\data}}
\newcommand{\prior}{\ensuremath{\prob (\model)}}

% distribution functions
\newcommand{\df}{f}
\newcommand{\dfs}[1]{\df_\mathrm{#1}}
\newcommand{\dfhalo}{\dfs{halo}}
\newcommand{\dfbulge}{\dfs{bulge}}
\newcommand{\dfdisk}{\dfs{disk}}

\newcommand{\E}{\ensuremath{E}}
\newcommand{\Lz}{\ensuremath{L_Z}}
\newcommand{\Ez}{\ensuremath{E_Z}}

% model parameters
\newcommand{\halo}{\mathrm{h}}
\newcommand{\bulge}{\mathrm{b}}
\newcommand{\disk}{\mathrm{d}}
\newcommand{\mass}{\ensuremath{M}}

\newcommand{\ah}{\ensuremath{a_\halo}}
\newcommand{\vh}{\ensuremath{\sigma_\halo}}
\newcommand{\nh}{\ensuremath{n_\halo}}
\newcommand{\Md}{\ensuremath{\mass_\disk}}
\newcommand{\Rd}{\ensuremath{a_\disk}}
\newcommand{\vb}{\ensuremath{\sigma_\bulge}}
\newcommand{\Rb}{\ensuremath{a_\bulge}}
\newcommand{\nb}{\ensuremath{n_\bulge}}

% change of coords basis
\newcommand{\uvec}[1]{\ensuremath{\vector{e}_{#1}}}
\newcommand{\ex}{\uvec{x}}
\newcommand{\ey}{\uvec{y}}
\newcommand{\ez}{\uvec{z}}
\newcommand{\epx}{\ensuremath{\uvec{x}^\prime}}
\newcommand{\epy}{\ensuremath{\uvec{y}^\prime}}
\newcommand{\epz}{\ensuremath{\uvec{z}^\prime}}
\newcommand{\posand}{\ensuremath{\vector{r}_\mathrm{M31}}}
\newcommand{\R}{\ensuremath{\matrix{R}}}
\newcommand{\Rx}{\ensuremath{\R_x}}
\newcommand{\Ry}{\ensuremath{\R_y}}
\newcommand{\Rz}{\ensuremath{\R_x}}
\newcommand{\Rrpq}{\ensuremath{\R_{rpq}}}

% velocity corrections
\newcommand{\vel}[1]{\ensuremath{\vector{v}_\mathrm{#1}}}
\newcommand{\vlos}{\ensuremath{v_\mathrm{los}}}
\newcommand{\vt}{\ensuremath{\vector{v}_\mathrm{t}}}
\newcommand{\vrel}[2]{\ensuremath{\vector{v}_{\mathrm{#1},\mathrm{#2}}}}
\newcommand{\los}{\ensuremath{\hat{\vector{l}}}}
\newcommand{\losperp}{\ensuremath{\hat{\vector{l}}_\perp}}
\newcommand{\lossat}{\ensuremath{\hat{\vector{l}}_\mathrm{S}}}
\newcommand{\angsep}{\ensuremath{\Phi}}
\newcommand{\posang}{\ensuremath{\Theta}}
\newcommand{\posangperp}{\ensuremath{\Theta_\perp}}
\newcommand{\vw}{\ensuremath{v_\mathrm{W}}}
\newcommand{\vn}{\ensuremath{v_\mathrm{N}}}

% random other stuff
\newcommand{\Rband}{\emph{R}-band}
\newcommand{\Kband}{\emph{R}-band}

% units
\newcommand{\unit}[1]{\,\mathrm{#1}}
\newcommand{\kpc}{\unit{kpc}}
\newcommand{\kms}{\unit{km\,s^{-1}}}

\begin{document}

\title{The Mass of \andromeda: A Fully Self-Consistent Dynamical Model}

% \slugcomment{To be submitted for publication in ApJ}

\author{Daniel Foreman-Mackey\altaffilmark{\ref{affil:email},
                \ref{affil:nyu},\ref{affil:queens}}
    \& Lawrence M. Widrow\altaffilmark{\ref{affil:queens}}}

\altaffiltext{1}{\label{affil:email}
    To whom correspondence should be addressed: \url{danfm@nyu.edu}}
\altaffiltext{2}{\label{affil:nyu}
    Center for Cosmology and Particle Physics, Department of Physics,
    New York University,
    4 Washington Place, New York, NY 10003, USA}
\altaffiltext{3}{\label{affil:queens}Department of Physics, Engineering Physics, and Astronomy,
       Queen's University, Kingston, ON K7L 3N6, Canada}

\begin{abstract}

    Due to it's proximity, there is a wealth of kinematic measurements of M31
    at all scales. These datasets provide an extremely detailed snapshot of the
    dynamics of the galaxy at both dark matter and baryon dominated scales. We
    compile a catalog of satellite galaxy kinematics in the outer reaches of the
    dark matter halo and the HI rotation curve and (K-, I-band) surface brightness
    profile at smaller radii.
    To combine these heterogeneous datasets into a self-consistent probabilistic
    analysis, we develop a multicomponent, axisymmetric, semi-analytic, dynamical
    generative galaxy model and marginalize over the large experimental
    uncertainties and many nuisance parameters to infer the structural parameters
    of M31.
    We find that when we only consider so-called ``non-informative''
    parameterizations of our prior knowledge about the nuisance parameters, the
    data does not contain enough information to provide useful constraints on
    the physical galaxy parameters. Instead, by using physically motivated
    prior information about the stellar mass-to-light ratios, we find excellent
    constraints M31's structural parameters (TODO: what does this mean and how do
    we quantify it). TODO: describe the galaxy models here qualitative and
    quantitatively.

    To simultaneously marginalize and sample the posterior distribution, we
    we generalize and parallelize the affine invariant Markov chain Mont Carlo
    algorithm developed by Goodman \& Weare. This algorithm has only one tuning
    parameter --- instead of the $D\,(D+1)/2$ hyperparameters that must
    be fine-tuned using the standard methods --- making an expensive ``burn-in''
    phase unnecessary. This algorithm is especially useful in problems (like this
    one) where the likelihood evaluations require an expensive numerical
    computation.

\end{abstract}

\keywords{
methods: data analysis
---
methods: statistical
---
Local Group
---
galaxies: kinematics and dynamics
---
galaxies: structure
}

% -------- BODY --------

\defcitealias{ew}{EW00}
\defcitealias{vdm}{vdMG08}

\section{Introduction}

\begin{itemize}

\item Use of satellite galaxies as a probe of the Galactic potential
has had a long history dating back to Hartwick and Sargent (1978) (see
Little and Tremaine for an historical overview and further
references).  More recently, the methods have been applied to M31 (the
only other galaxy where accurate distances and velocities, relative to
the host galaxy, have been observed.  Early methods used simple mass
estimators.

\item Little and Tremaine introduced the method of Bayesian inference
to the problem for the Milky Way.  BI allows one to estimate
confidence intervals.  Furthermore, Bayesian inference allows one to
properly propagate errors in measurements (marginalization).  As well,
LT modelled the tracer population using a distribution function.

\item Further step was taken by Kochanek (1996) who used a more ``realistic''
and flexible distribution function (Jaffe model).

\item Evans and Wilkinson (1999) present the first such analysis of
M31.  Most of the previous estimates of M31 mass were based either on
the rotation curve, which extends to only 30 kpc.  Courteau and van
den Bergh (1999) used simple mass estimators for a sample of seven
satellites.

\item The basic idea of the Bayesian treatment is to build a
generative model for both the distribution function of the tracers and
the gravitational potential and then marginalize over the tracer DF.

\end{itemize}

\section{The Data}

\paragraph{Satellite galaxy kinematics}

Table \ref{tab:satdata} lists the relevant information for the 23 satellite
galaxies in our sample.  This sample was constructed by supplementing the
catalogues from \citet{Evans:2000a} and \cite{McConnachie:2006} with more
recently discovered dwarf galaxies and updated distance and velocity measurements.
We restrict our sample to galaxies with measurements of both line-of-sight
velocity and distance.

\object[And IX]{Andromeda IX} and \object[And X]{X} were discovered by
\citet{Zucker:2004} and \citet{Zucker:2007} respectively in the Sloan Digital
Sky Survey \citep[\sdss;][]{York:2000} and subsequently observed spectroscopically
by \citet{Collins:2010} and \citet{Kalirai:2010}.  \citet{Kalirai:2010} also
refined the distance and line-of-sight velocity measurements for
\object[And I]{Andromeda I}, \object[And II]{II}, \object[And III]{III},
\object[And VII]{VII} and \object[And XIV]{XIV}.  \object[And XI]{Andromeda XI},
\object[And XII]{XII} and \object[And XIII]{XIII} were discovered by
\citet{Martin:2006} in a wide-field photometric survey of M31's outer halo
their distances and line-of-sight velocities were measured by
\citet{Collins:2010}.  \object[And XIV]{Andromeda XIV} was discovered by
\citet{Majewski:2007} and followed up by \citet{Kalirai:2010}.  The two most
recently discovered satellites with both line-of-sight velocity and distance
measurements -- \object[And XV]{Andromeda XV} and \object[And XVI]{XVI} -- were
both discovered by \citet{Ibata:2007} in the first year data from the
Pan-Andromeda Archaeological Survey (\pandas) and followed up by \citet{Letarte:2009}.

The distance measurement uncertainties for Andromeda IX, XI, XII and XIII are
extremely asymmetric about the mode and therefore present an interesting
inference problem.  In the limit of symmetric uncertainties, we can follow the
standard procedure of modeling the probability distribution function (PDF) of
the measurement as a Gaussian random variable.  This model breaks down, however,
when the uncertainties are significantly asymmetric.  Without samples from the PDF
of the original distance measurement, we are free to assume a parametric model
for the measurement PDF.  For simplicity, we choose a generalization of the
Gaussian distribution the skew-normal distribution (appendix \ref{sect:sn})
specified by one extra parameter.  Table \ref{tab:sn} lists the skew-normal
parameterizations of the asymmetric distance measurements.


\paragraph{Rotation curve}

\label{sect:rcdata}

The rotation curve of M31 has been used extensively for mass modeling and it is
credited as one of the earliest arguments for dark matter \citep{Rubin:1970}.
Most measurements of gas kinematics are more than 20 years old and they are all
of quite low resolution.  \citet{Chemin:2009} recently published a high resolution
study of the HI kinematics in M31 (figure \ref{fig:rc_data}).  \Fig{rc_data}
shows the observed gas speed, position angle and inclination from the authors'
tilted-ring analysis.  We can see that the circular velocity profile in this
figure has significant kinematic structure not due to a smooth, axisymmetric
density distribution.  Comparing this to the plots of position angle and inclination
elucidates this even further.  In the inner $R_\mathrm{p} \lesssim 30^\prime \sim 7 \kpc$
and outer $R_\mathrm{p} \gtrsim 130^\prime \sim 30 \kpc$, the disk is much less
inclined than the median inclination and there is also significant scatter.
It is likely that these features are due to departures from axisymmetry and
non-circular motions (Cite Spekkens?).  \citet{Chemin:2009} argue that their
quoted uncertainties should account for some departure from axisymmetry but if
the velocity field is highly non-circular then it is unsuitable for dynamical
analysis at a na\"ive level.  With this in mind,

% Because of the uncertainty and scatter in the inner and outer regions of these plots, we only fit the data in the radial range $30^\prime \le R \le 130^\prime$.

(DFM: elaborate on the problems with this data)


\paragraph{Surface brightness profile}

Larry: fill this section in...

% The kinematic data provides a strong constraint on the mass distribution of the
% galaxy but we should also recover the observed surface brightness profile for
% reasonable mass-to-light ratios of the disk and bulge.  For our observational
% constraint, we consider the classic measurement of the \Rband surface brightness
% profile from \citet{Walterbos:1987,Walterbos:1988}.  We choose to use the \Rband
% as opposed to a bluer band since it is less affected by dust and more sensitive
% to the low-mass stars that dominate the mass distribution
% \citep[e.g.][]{Binney:1998}.

\section{The Model}

To numerically generate the observations described in the previous section, we
use the model originally proposed by \citet{Kuijken:1995} and further improved
by \citet{Widrow:2008}.  The produced galaxy models consist of three axisymmetric,
dynamically distinct components --- a disk, a bulge and a dark matter halo.
By construction, they are simultaneously equilibrium solutions to the collisionless
Boltzmann equation and the self-gravitational Poisson equation \citep{Binney:2008}.
The phase-space distribution of particles in the model is fully specified by a
multi-component distribution function of three integrals of motion: the energy
\E, the angular momentum about the symmetry axis \Lz, and the vertical kinetic
energy of the disk stars \Ez
\begin{equation}
    \eqlabel{df1}
    \df (\E,\Lz,\Ez) = \dfhalo (\E) + \dfbulge (\E) + \dfdisk (\E,\Lz,\Ez).
\end{equation}

This distribution function is iteratively fine-tuned to match astrophysically
motivated density profiles for each component as described by \citet{Widrow:2008}.
The target densities that we choose are:
\begin{itemize}
    \item{an Einasto (CITATION?) halo,}
    \item{a de-projected S\'ersic profile \citep{Prugniel:1997,Terzic:2005}, and}
    \item{an exponential thin disk of mass \Md.}
\end{itemize}
It is important to note that although the target density profiles for the halo
and bulge are spherically symmetric, these components in the final model are
somewhat flattened by the disk potential \citep{Widrow:2008}.  In total, the
galaxy model is fully specified by nine physical parameters.  These and several
other input values are listed in \tab{params} alongside their assumed priors.




\section{The Method}

Since we would like to infer what basic structural parameters of M31 are consistent
with observation, we must Bayes' theorem to reconcile the model and the data.
In more precise terms, we wish to infer the probability, \posterior, of a set
of model parameters, \model, given the data, \data.  Conversely, what we can
easily construct is the likelihood of the data given a set of model parameters,
\likelihood.  Bayes' theorem allows us to relate these two quantities as
\begin{equation}
    \posterior \propto \prior \times \likelihood
\end{equation}
where \prior quantifies any prior knowledge we have about the model parameters.
To be more rigorous, all quantities in the above equation are also conditional
on our assumption that the model is applicable but we will not include this
explicitly for simplicity of notation.

In the subsequent sections, we outline the mathematical construction of \likelihood
and then discuss our choice of priors.

\paragraph{Satellite galaxy kinematics}

\citet{Sales:2007b} analyzed the distribution of satellite galaxies formed in
$N$-body/gas-dynamical simulations including dark matter, gas and stars and found
that within the statistical errors, the satellites follow the same density
distribution as the dark matter.  Following this result, we assume that the
satellite galaxies are sampled directly from the dark matter distribution.
The true distribution of satellites is probably more complex than this simple
assumption \citep[e.g.][]{Ibata:2007}, however, given the sparseness of the
dataset, this is a necessary assumption.

Under this assumption, the phase space distribution function of satellite galaxies
is identically $\dfhalo (\E)$ (the halo DF from \eq{df1}) with
$\E = \Psi (s,z) + \left ( \vlos^2 + \vt^2 \right )/2$ where
$(s,z)$ are the M31-centric cylindrical coordinates of the satellite,
$\vlos \equiv v_\mathrm{S,A}^\mathrm{los}$ is given by \eq{vdm2}
and $v_\mathrm{t}$ is the unknown satellite velocity in the plane of the sky.

The positions and velocities of the satellite galaxies in our sample are all
measured in the heliocentric rest frame centered on the Sun but our model generates
points in a coordinate system centered on M31 in phase space. We must, therefore,
transform the observations into the rest frame of M31 in order to make any
comparisons to the model. The position space transformation has the form
\citep{Metz:2007}
\begin{equation}
    \bvec{r^\prime} = \bvec{R} (\bvec{r} - \bvec{r_\mathrm{M31}})
\end{equation}
where \bvec{r^\prime} and \bvec{r} are the position vectors of the satellite in
Andromeda-centric and heliocentric coordinates respectively, \bvec{r_\mathrm{M31}}
is the observed position of Andromeda and \bvec{R} is given by \eq{fullrotation}.
In appendix \ref{sect:coords}, we summarize the construction of this
transformation. Since the distance to M31 is a parameter in our model, we must
actively perform this transformation for every sample in our MCMC analysis.


\section{The Likelihood Function}

\paragraph{Satellite galaxy kinematics}

Based on the considerations of the previous section, the likelihood of observing
satellite $i$ at its measured position in phase space given a specific model $\Theta$ is
\begin{equation}
    \eqlabel{satprob1}
    \mathrm{p}_\mathrm{sat}(\mathbf{X}_i | \Theta,I) =
    \left . \int d^6 \mathbf{X} \, S(\mathbf{X} | \mathbf{X}_i) f_\mathrm{h} (E)
    \right /\int d^6 \mathbf{X} \, f_\mathrm{h} (E)
\end{equation}
where $S(\mathbf{X} | \mathbf{X}_i)$ is the distribution of the measured quantities
(including the uncertainties).  For our entire sample of satellite galaxies, three
components of position space (right ascension, declination and distance) and one
component of velocity space (line-of-sight) have been measured while the two
transverse velocity components are unknown.  Therefore, the first integral in
\eq{satprob1} can be rewritten
\begin{eqnarray}
    \eqlabel{satprob2}
    \int d^6 \mathbf{X} \, S(\mathbf{X} | \mathbf{X}_i) f_\mathrm{h} (E)
    & = & 2 \pi \, \int_{-\infty}^{\infty} dD \int_{-\infty}^{\infty} dv_\mathrm{obs}
    \int_0 ^{v_\mathrm{max}} v_{\mathrm{t}} \, dv_{\mathrm{t}}
    g(D,v_\mathrm{obs},v_\mathrm{t})
\end{eqnarray}
where
\begin{equation}
    g (D,v_\mathrm{obs},v_\mathrm{t})
    = \tilde{\mathcal{N}} (D | \xi_i, \omega_i, \alpha_i) \,
    \mathcal{N} (v_\mathrm{obs} | v_{\mathrm{obs},i},\delta v_{\mathrm{obs},i})\,
    f_\mathrm{h} (E)  ,
\end{equation}
$v_\mathrm{max} \equiv \sqrt{2 \Psi - v_\mathrm{los}^2}$ is the maximum transverse
component of the satellite's velocity;
$v_\mathrm{obs} \equiv \mathbf{v}_\mathrm{S} \cdot \hat{l}_\mathrm{S}$ is the
observed line-of-sight velocity;
$\mathcal{N} (v_\mathrm{los}; v_\mathrm{los,i},\delta v_\mathrm{los,i})$ is the
Gaussian PDF; and  $\tilde{\mathcal{N}} (D; \xi_i, \omega_i, \alpha_i)$ is the
skew-normal parameterization of the distance measurement.  The angular position
integrals are simply delta functions (because the measurement uncertainty is
negligible) so they have been omitted \eq{satprob2}.

We evaluate the integration in \eq{satprob2}  numerically using Monte
Carlo integration with importance sampling for $D$ and $v_\mathrm{obs}$ and
Simpson's rule for $v_{\mathrm{t}}$.  We calculate the full log-likelihood of the
satellite data given the model by taking the logarithm of \eq{satprob2}
and summing over the set of satellites
\begin{equation}
    \ln \mathrm{p}_\mathrm{s} (\mathbf{X}| \Theta, I)
    = \sum_i \ln \mathrm{p}_\mathrm{s} (\mathbf{X}_i | \Theta,I).
\end{equation}
Similarly, the normalization due to the second integral in \eq{satprob1} is
\begin{equation}
    \eqlabel{satnorm}
    N_\mathrm{s} \times
        \ln \left [ \int d^6 \mathbf{X} \, f_\mathrm{h} (E) \right ] =
        N_\mathrm{s} \ln M_\mathrm{h}
\end{equation}
where $N_\mathrm{s}$ is the number of satellites in the sample and the integral
is the total halo mass, by definition.  In practice, we only consider the mass
within some limiting radius $r_\mathrm{max}$ where $r_\mathrm{max}$ is approximately
the distance of the furthest satellite from M31 (the assumed edge of the survey).
We choose a constant $r_\mathrm{max} = 550$ kpc but the results are insensitive
to this choice.

\paragraph{Rotation curve}

From the gravitational potential of our full galaxy model, we can calculate the
expected circular velocity $v_\mathrm{m}(r)$ at a given radius $r$ in the disk.
We compare this value to the observed value $v_i$ at radius $r_i$ assuming
uncorrelated Gaussian uncertainties
$\delta v_i = \sqrt{\delta v_{0,i}^2 + \epsilon_\mathrm{RC}}$ where
$\delta v_{0,i}^2$ is the published uncertainty and $\epsilon_\mathrm{RC}$ is a
free parameter --- called ``jitter'' --- that accounts for noise in the data and
allows the true uncertainties to be inferred.  For simplicity, we assume that
$\epsilon_\mathrm{RC}$ is a constant with radius.  The log-likelihood of the
observed values given a particular model specified by the set of parameters
$\mathbf{\theta}$ is
\begin{equation}
    \eqlabel{rclike}
    \ln \mathrm{p}_\mathrm{rc} (\{ r_i, v_i, \delta v_i \}| v_\mathrm{m},
    \epsilon_\mathrm{RC}) = - \sum_i \left [ \frac{[v_\mathrm{m}(r_i) - v_i]^2}
        {2 \delta v_i ^2} - \frac{1}{2} \ln (2 \pi \delta v_i ^2) \right ].
\end{equation}
This is simply the standard result that $p \propto e^{-\chi^2/2}$ but it is
important to recognize that the normalization coefficient is a function of
$\epsilon_\mathrm{RC}$ so it will not remain constant across models.
Intuitively, the second term on the righthand side of \eq{rclike} penalizes
models with large values of $\epsilon_\mathrm{RC}$ for a lack of specificity.

\paragraph{Surface brightness profile}

As for the rotation curve, we assume Gaussian uncertainty and calculate the
likelihood of the observed surface brightness profile given our model.  In
particular, we assume that the uncertainty can be parameterized by a constant
variance $\epsilon_\mu$ at all radii.  We then calculate the surface brightness
profile of our model $\mu_R (r_i)$ in the $R$-band assuming constant
mass-to-light ratios for the disk and bulge and compare it to the data
$\mu_i$ (at radius $r_i$).  The likelihood of the data is therefore
\begin{equation}
    \eqlabel{sbpprob}
    \ln \mathrm{p}_\mathrm{sbp} (\{ r_i, \mu_i \} | \mu_R,\epsilon_\mu )
    = - \frac{1}{2} \sum_i \left [ \frac{\left [ \mu_R (r_i) - \mu_i \right ]^2 }
        {\epsilon_\mu} + \ln (2 \pi \epsilon_\mu) \right ]
\end{equation}
analogous to \eq{rclike}.



\section{Priors}

\label{sect:priors}

\Tab{params} lists all the model parameters and their assumed priors.

For any dimensioned parameter that can vary from zero to infinity, it is best to
assume a logarithmic prior, especially if there is large uncertainty in the
physically allowed range \citep{Gregory:2005}.  The logarithmic prior enforces
scale invariance and it is implemented by sampling in the logarithm of the parameter.
To this end, we adopt uniform priors on the logarithm of the scale parameters in
our model unless a more informative prior is available.

For the $R$-band mass-to-light ratios of the disk and bulge, we set a conservative
hard upper limit of $\Upsilon_R \le 10 \, M_\odot /\mathcal{L}_{\odot,R}$ but
the data provides a much tighter constraint.  For the components of the systematic
transverse velocity, we assume a uniform prior on $v_\mathrm{W}$ and $v_\mathrm{N}$
but set a hard upper limit on the total transverse velocity of
$\sqrt{v_\mathrm{W}^2+v_\mathrm{N}^2} < 1000$ km s$^{-1}$ for computational purposes.

Throughout, we adopt a Gaussian prior on the distance with mean $785$ kpc and
standard deviation $25$ kpc \citep{McConnachie:2005,McConnachie:2006}.  Similarly,
for the systematic heliocentric line-of-sight velocity, we adopt the value
$-301 \pm 1$ km s$^{-1}$ \citep{Courteau:1999}.

\section{Sampling the Posterior Probability Density}

Combining the likelihood functions constructed in the previous section and
applying Bayes' theorem, we have that the posterior probability of the model
parameters given the full data set is
\begin{equation}
    \eqlabel{fullposterior}
    \ln \mathrm{p} (\mathbf{\Theta} | \mathbf{X}, I) =
    \ln \mathrm{p} (\mathbf{\Theta} | I) +  \ln \mathrm{p}_\mathrm{sat}
    + \ln \mathrm{p}_\mathrm{rc}
    + \ln \mathrm{p}_\mathrm{sbp} + \mathrm{constant}
\end{equation}

Our generative model of the data is numerical and none of the marginalization
integrals are analytic. Therefore, to estimate the value of \eq{fullposterior}
and perform the marginalization integrals over nuisance parameters, we must
rely on sampling. This is a large computational challenge since every evaluation
of \eq{fullposterior} requires substantial expensive computation. To reduce the
runtime, we use an extremely efficient sampling algorithm that employs an affine
invariant proposal. Intuitively, affine invariance means that the algorithm is
insensitive to the aspect ratio of the density even if it is highly anisotropic.
Traditional MCMC algorithms require an extensive ``burn-in''.

In practice, we are actually interested in the marginalized form of \eq{fullposterior}
\begin{equation}
    \eqlabel{marginalization}
    p (\tilde{\boldsymbol{\Theta}} | \mathbf{X}) \propto
        p (\tilde{\boldsymbol{\Theta}}) \int
        p (\mathbf{X} | \tilde{\boldsymbol{\Theta}},\boldsymbol{\alpha}) \,
        p (\boldsymbol{\alpha}) \, \dd \boldsymbol{\alpha}
\end{equation}
where $\tilde{\boldsymbol{\Theta}}$ is the vector of physical parameters of
interest. The integral in
\eq{marginalization} is over a set of \emph{nuisance parameters} $\boldsymbol{\alpha}$
that are (generally) of no physical interest in the problem. The function
$p (\mathbf{X} | \tilde{\boldsymbol{\Theta}},\boldsymbol{\alpha}) \, p (\boldsymbol{\alpha})$
is not integrable analytically. Instead, since the likelihood function is only
numerical, the integration in \eq{marginalization} is actually very expensive
to compute. In this regime, the numerical integration algorithm of choice is
generally Markov chain Monte Carlo (MCMC).
Since the likelihood function is expensive to calculate, it is advantageous to
use a sampling algorithm that reduces the necessary number of likelihood
evaluations. This also precludes the use of second order methods (such as
hybrid/Hamiltonian Monte Carlo) that require the calculation of (numerical)
gradients of the likelihood function.

Most uses of MCMC in the astrophysics literature are based on slight modifications
to the Metropolis-Hastings (M-H) method \citep[e.g.][]{MacKay:2003}. Each step in
a M-H chain is proposed using a multivariate Gaussian centered on the current
position of the chain. Since each term in the covariance matrix of this proposal
distribution is an unspecified parameter, this method has $D\,[D+1]/2$ tuning parameters
(where $D$ is the dimension of the parameter space).  To make matters worse, the
performance of this sampler is very sensitive to the optimality of these tuning
parameters and there is no fool-proof method for choosing the values correctly.
As a result, many heuristic methods have been developed to attempt to determine
the optimal parameters in a data-driven way \citep[e.g.][]{Gregory:2005,Dunkley:2005,Widrow:2008}.
Unfortunately, these methods all require ``burn-in'' phases where shorter Markov chains
are sampled and the results are used to tune the hyperparameters.

\paragraph{The stretch move}

\citet{Goodman:2010} proposed an affine invariant ensemble sampling algorithm
informally called the ``stretch move''. This method involves simultaneously
evolving an ensemble of $K$ \emph{walkers} where the proposal distribution for one
walker $k$ is based on the current positions of the $K-1$ walkers in the
\emph{complementary ensemble}.

In practice, to update the position of a walker at position $\mathbf{X}_k$,
another walker $\mathbf{X}_j$ with $j \ne k$ is randomly chosen and then
a new position is proposed:
\begin{equation}
    \eqlabel{proposal}
    \mathbf{X}_k (t) \rightarrow \mathbf{Y} = \mathbf{X}_j + Z \, [\mathbf{X}_k (t) - \mathbf{X}_j]
\end{equation}
where $Z$ is a random variable drawn from a distribution $g(z)$.  It is clear that
if $g(z)$ satisfies
\begin{equation}
    g(z^{-1}) = z \, g(z),
\end{equation}
the proposal of \eq{proposal} is symmetric. In this case, the chain will satisfy
detailed balance if the proposal is accepted with probability
\begin{equation}
    \min \left \{ 1, Z^{n-1} \, \frac{p(\mathbf{Y})}{p(\mathbf{X}_k(t))} \right \}.
\end{equation}
This procedure is then repeated for each walker in the ensemble \emph{in series}.

\citet{Goodman:2010} advocate for a particular form of $g(z)$, namely
\begin{equation}
    g(z) \propto \left \{ \begin{array}{ll}
        \displaystyle\frac{1}{\sqrt{z}} & \mathrm{if}\, z\in \left [ \displaystyle\frac{1}{a}, a \right ], \\
        0 & \mathrm{otherwise}
    \end{array} \right .
\end{equation}
where $a$ is an adjustable scale parameter. For comparison, I will also use this
distribution with $a=2$ for all benchmarks in this paper.

\paragraph{The parallel stretch move}

It is tempting to na\"ively parallelize the stretch move algorithm by
simultaneously advancing each walker based on the state of the ensemble
instead of evolving the walkers in series. Unfortunately, this would no longer
satisfy detailed balance. Instead, if we split the ensemble into two ensembles
(labeled \emph{blue} and \emph{red} for convenience) and simultaneously update
all the blue walkers --- using the stretch move procedure --- based on the
positions of \emph{only the
red walkers} then the outcome is a valid step for each of the walkers. Then,
the red walkers are advanced based only on the positions in the blue ensemble.

The performance of this method --- quantified by the autocorrelation time ---
is comparable to the traditional stretch move algorithm but the fact that one
can now take advantage of generic parallelization makes this generalization
extremely powerful.

\section{Results}

Our first results come from including all the datasets discussed above and
sampling the full posterior function considering only the na\"ive priors
discussed above. As we will argue below, this dataset is not sufficient
for constraining the model and the results are not acceptable.

For this and all other experiments, we used 100 walkers for our sampling
algorithm and (DFM: how many?) 5000 steps in the chains. After the 1000
step burn-in, this yields $4\times10^6$ samples from the posterior function.
(DFM: do we?) For display purposes, we resample each chain by 100 before
plotting.

\paragraph{Fitting just the satellites}

(do we want this?)

\paragraph{Fits to the data}

The results of the na\"ive chains provide excellent qualitative and
quantitative fits to the rotation curve and surface brightness profile
as seen in figures (DFM: add figures). It is clear from the disk-bulge-halo
decomposition in figure (DFM: rotation curve figure) that the preferred
model has a maximal disk. This result is, of course, somewhat contentious
(DFM: add some citations here). Even more problematic are the relative
stellar mass-to-light ratios of the disk and bulge. It is generally expected
that these two quantities should be of the same order with the bulge mass-
to-light ratio being somewhat larger. Our preferred model does not satisfy
this expectation (figure DFM: add figure) and instead, the mass-to-light
ratio of the disk is significantly larger than that of the bulge. Another
problem with these results is visible in figure (DFM: mass plot). This
figure shows the constraints on the mass profile of M31 overplotted with
several quantities from the literature. While the constraint at smaller
radii is consistent with previous measurements, the values at large radii
are inconsistent and also unphysical.

(LW: Comment on stability here)


\subsection{Theoretical mass-to-light ratios from stellar population
    synthesis}

A further constraint is clearly necessary to make this procedure more
meaningful. We apply a prior on the mass-to-light ratio of the disk based
on the stellar population models of \citet{Bell:2003}.

(LW: fill in the comments about the derivation of the M/L prior)

We perform the same analysis as in the previous section with the added
informative uniform prior on the disk mass-to-light ratio. This physically
motivated addition is enough to mostly solve the problems listed above. By
construction, the disk mass is reduced, yielding a more dynamically stable
solution but more interestingly, it produces a mass profile that is physically
reasonable over the full range of radii.

\subsection{Constraints on the physical galaxy parameters}



\subsection{Transverse velocity}

As discussed previously, the line-of-sight velocities of the satellites also
constrain the transverse motion of M31 on the sky through \eq{vdm2}. The
distribution of proper motion values that we find to be consistent with the
data is shown in \fig{vt}. Our results are consistent with the values found
by \citet{vdm} in their simpler modeling procedure. Our constraint on
M31's transverse motion is
\begin{equation}
    \bvec{v_T} =
    \left ( \begin{array}{c}
        v_W \\ v_N
    \end{array} \right ) =
    \left ( \begin{array}{c}
        -35 \\ 32
    \end{array} \right ) \kms
\end{equation}
with covariance
\begin{equation}
    \mathrm{cov} (\bvec{v_T}) = \left ( \begin{array}{cc}
        17848 & 1459 \\
        1459  & 4557
    \end{array}\right ) \unit{km^2\, s^{-2}}.
\end{equation}
The eigenvectors of this covariance tensor are shown in \fig{vt} and the
uncertainty in this rotated space (the square root of the eigenvalues) is
$(134, 66) \kms$. These error bars are on the same order as the
uncertainties on the comparable result from \citet{vdm}. This is somewhat
surprising because our galaxy model is much more general and we
would expect that the procedure of properly marginalizing over all the
other nuisance parameters should reduce the specificity of the result.
In this study, we did not include and constraints on the transverse motion
of M33 and (DFM: which other galaxy?) from (CITATION). If we were to include
these measurements, we expect that we would find better constraints on this
proper motion.




\section{Discussion}

We have developed a fully self-consistent probabilistic method of inferring
the dynamical parameters of a multi-component disk galaxy using heterogeneous
datasets at all radial ranges. Our method properly marginalizes over missing
data and makes full use of the uncertainties. In this paper, we applied our
technique to kinematic tracers in M31 but it could also be useful in the
Milky Way or even galaxies outside of the local group. In conjunction with
our method we introduce a novel MCMC algorithm that is extremely useful for
problems (such as this one) where each evaluation of the likelihood function
is computationally expensive.

To evaluate the probability of the data given a particular set of model
parameters, we generate (numerically) a semi-analytic, fully self-consistent
phase space distribution function for each dynamically interacting component
of the galaxy. Then, by marginalizing over all the

Summary

Comparison with previous work








% \subsection{Stability Considerations}

% TODO: Do we actually want this section?

% % \citet{Widrow:2008} presented a numerical the stability of a galactic disk is very sensitive to the relative disk mass.

% Near-infrared imaging and kinematic arguments suggest that M31 has a (weak?) bar \citep{Beaton:2007,Saglia:2010}, presenting a strong departure from axisymmetry in the central regions of the galaxy.  Many attempts have been made to confront this complication when modeling the Galaxy by constructing and numerically evolving axisymmetric N-body models and then comparing to observations \citet{Ostriker:1973,Sellwood:1985,Fux:1997,Widrow:2008}.  The most important conclusion for our purposes is that the stability of a galactic disk is very sensitive to the relative disk mass.  It has been shown \citep{Debattista:2000,Widrow:2008} that the ratio of total circular velocity $v_\mathrm{tot}$ and the contribution of an isolated disk $v_\mathrm{d}$ measured at 2.2 disk scale lengths
% \begin{equation}
%     X^\prime = \frac{v_\mathrm{tot}^2 (2.2\,R_\mathrm{d})}{v_\mathrm{d}^2 (2.2\,R_\mathrm{d})}
% \end{equation}
% is a good proxy for estimating the stability of a disk.  In particular, \citet{Widrow:2008} showed that this parameter is equivalent to the Goldreich-Tremaine $X$ stability parameter \citep{Goldreich:1978,Goldreich:1979} with negligible scatter.  The disk density profile of a galaxy model with $X^\prime \lesssim 3$ (Larry:  is this number right?) becomes extremely distorted when evolved numerically\footnote{See http://www.cita.utoronto.ca/~dubinski/DynamicalBlueprints/ for a visualization.} and the resulting disk is qualitatively very different than observed.

% Although the parameter $X^\prime$ doesn't provide a rigorous, quantitative constraint on our models, we can consider a grid of lower limits on $X^\prime$ to determine its effect on our inference.  We impose a cubic prior on $X^\prime$ varying smoothly from zero at some $X_\mathrm{min}$ to unity at $X_\mathrm{min}+\Delta X$.

% DFM: Right now I've only done $(X_\mathrm{min},\Delta X) = (3,1)$ should also do: $X_\mathrm{min} = 2,2.5,3.5,4$? or something like that.




% \subsection{Proposed Model}

% DFM: Argue for a specific model here ?



% \section{Conclusions}




% -------- ACKNOWLEDGEMENTS --------

\acknowledgments It is a pleasure to thank Terry Bridges (Queen's),
St\'ephane Courteau (Queen's), David Hanes (Queen's) and David W. Hogg (NYU)
for insightful discussions and suggestions. D.F.M. was partially supported
by NASA (grant WHICH GRANT).

(LARRY: any grants for you?)

This research made use of the \project{Python} programming language and the
open-source modules \project{scipy}, \project{numpy}, \project{matplotlib}
and \project{emcee}\footnote{\url{http://danfm.ca/emcee}}.  It also
made extensive use of A. Collette's HDF5 for Python (\project{h5py}). All
code and data used in this project are available from the authors upon
request.

% -------- REFERENCES --------

\bibliographystyle{apj}

\begin{thebibliography}{}

\bibitem[{{Azzalini} \& {Dalla Valle}(1996)}]{Azzalini:1996}
{Azzalini}, A., \& {Dalla Valle}, A. 1996, Biometrika, 83, 715

\bibitem[{{Bahcall} \& {Tremaine}(1981)}]{Bahcall:1981}
{Bahcall}, J.~N., \& {Tremaine}, S. 1981, \apj, 244, 805

\bibitem[{{Bell} {et~al.}(2003){Bell}, {McIntosh}, {Katz}, \&
  {Weinberg}}]{Bell:2003}
{Bell}, E.~F., {McIntosh}, D.~H., {Katz}, N., \& {Weinberg}, M.~D. 2003, \apjs,
  149, 289

\bibitem[{{Binney} \& {Tremaine}(2008)}]{Binney:2008}
{Binney}, J., \& {Tremaine}, S. 2008, {Galactic Dynamics: Second Edition}
  (Princeton University Press)

\bibitem[{{Braun}(1991)}]{Braun:1991}
{Braun}, R. 1991, \apj, 372, 54

\bibitem[{{Chemin} {et~al.}(2009){Chemin}, {Carignan}, \&
  {Foster}}]{Chemin:2009}
{Chemin}, L., {Carignan}, C., \& {Foster}, T. 2009, \apj, 705, 1395

\bibitem[{{Collins} {et~al.}(2010){Collins}, {Chapman}, {Irwin}, {Martin},
  {Ibata}, {Zucker}, {Blain}, {Ferguson}, {Lewis}, {McConnachie}, \&
  {Pe{\~n}arrubia}}]{Collins:2010}
{Collins}, M.~L.~M., {et~al.} 2010, \mnras, 407, 2411

\bibitem[{{Courteau} \& {Rix}(1999)}]{Courteau:1999a}
{Courteau}, S., \& {Rix}, H.-W. 1999, \apj, 513, 561

\bibitem[{{Courteau} \& {van den Bergh}(1999)}]{Courteau:1999}
{Courteau}, S., \& {van den Bergh}, S. 1999, \aj, 118, 337

\bibitem[{{de Vaucouleurs}(1958)}]{de-Vaucouleurs:1958}
{de Vaucouleurs}, G. 1958, \apj, 128, 465

\bibitem[{{de Vaucouleurs} {et~al.}(1991){de Vaucouleurs}, {de Vaucouleurs},
  {Corwin}, {Buta}, {Paturel}, \& {Fouque}}]{de-Vaucouleurs:1991}
{de Vaucouleurs}, G., {de Vaucouleurs}, A., {Corwin}, Jr., H.~G., {Buta},
  R.~J., {Paturel}, G., \& {Fouque}, P. 1991, {Third Reference Catalogue of
  Bright Galaxies}, ed. {de Vaucouleurs, G., de Vaucouleurs A. Corwin H.~G. Jr.
  Buta R.~J. Paturel G. \& Fouque P.}

\bibitem[{{Deharveng} \& {Pellet}(1975)}]{Deharveng:1975}
{Deharveng}, J.~M., \& {Pellet}, A. 1975, \aap, 38, 15

\bibitem[{{Dunkley} {et~al.}(2005){Dunkley}, {Bucher}, {Ferreira}, {Moodley},
  \& {Skordis}}]{Dunkley:2005}
{Dunkley}, J., {Bucher}, M., {Ferreira}, P.~G., {Moodley}, K., \& {Skordis}, C.
  2005, \mnras, 356, 925

\bibitem[{{Evans} \& {Wilkinson}(2000)}]{ew}
{Evans}, N.~W., \& {Wilkinson}, M.~I. 2000, \mnras, 316, 929

\bibitem[{{Evans} {et~al.}(2000){Evans}, {Wilkinson}, {Guhathakurta}, {Grebel},
  \& {Vogt}}]{Evans:2000a}
{Evans}, N.~W., {Wilkinson}, M.~I., {Guhathakurta}, P., {Grebel}, E.~K., \&
  {Vogt}, S.~S. 2000, \apjl, 540, L9

\bibitem[{{Evans} {et~al.}(2003){Evans}, {Wilkinson}, {Perrett}, \&
  {Bridges}}]{Evans:2003}
{Evans}, N.~W., {Wilkinson}, M.~I., {Perrett}, K.~M., \& {Bridges}, T.~J. 2003,
  \apj, 583, 752

\bibitem[{Goodman \& Weare(2010)}]{Goodman:2010}
Goodman, J., \& Weare, J. 2010, Comm. App. Math. Comp. Sci., 5, 65

\bibitem[{{Gottesman} \& {Davies}(1970)}]{Gottesman:1970}
{Gottesman}, S.~T., \& {Davies}, R.~D. 1970, \mnras, 149, 263

\bibitem[{{Gregory}(2005)}]{Gregory:2005}
{Gregory}, P.~C. 2005, {Bayesian Logical Data Analysis for the Physical
  Sciences: A Comparative Approach with `Mathematica' Support}, ed. {Gregory,
  P.~C.} (Cambridge University Press)

\bibitem[{{Hartwick} \& {Sargent}(1974)}]{Hartwick:1974}
{Hartwick}, F.~D.~A., \& {Sargent}, W.~L.~W. 1974, \apj, 190, 283

\bibitem[{{Ibata} {et~al.}(2007){Ibata}, {Martin}, {Irwin}, {Chapman},
  {Ferguson}, {Lewis}, \& {McConnachie}}]{Ibata:2007}
{Ibata}, R., {Martin}, N.~F., {Irwin}, M., {Chapman}, S., {Ferguson}, A.~M.~N.,
  {Lewis}, G.~F., \& {McConnachie}, A.~W. 2007, \apj, 671, 1591

\bibitem[{{Kalirai} {et~al.}(2010){Kalirai}, {Beaton}, {Geha}, {Gilbert},
  {Guhathakurta}, {Kirby}, {Majewski}, {Ostheimer}, {Patterson}, \&
  {Wolf}}]{Kalirai:2010}
{Kalirai}, J.~S., {et~al.} 2010, \apj, 711, 671

\bibitem[{{Klypin} {et~al.}(2002){Klypin}, {Zhao}, \& {Somerville}}]
    {Klypin:2002}
{Klypin}, A., {Zhao}, H., \& {Somerville}, R.~S.,
2002, \apj, 573, 597

\bibitem[{{Kuijken} \& {Dubinski}(1995)}]{Kuijken:1995}
{Kuijken}, K., \& {Dubinski}, J. 1995, \mnras, 277, 1341

\bibitem[{{Lee} {et~al.}(2008){Lee}, {Hwang}, {Kim}, {Park}, {Geisler},
  {Sarajedini}, \& {Harris}}]{Lee:2008}
{Lee}, M.~G., {Hwang}, H.~S., {Kim}, S.~C., {Park}, H.~S., {Geisler}, D.,
  {Sarajedini}, A., \& {Harris}, W.~E. 2008, \apj, 674, 886

\bibitem[{{Letarte} {et~al.}(2009){Letarte}, {Chapman}, {Collins}, {Ibata},
  {Irwin}, {Ferguson}, {Lewis}, {Martin}, {McConnachie}, \&
  {Tanvir}}]{Letarte:2009}
{Letarte}, B., {et~al.} 2009, \mnras, 400, 1472

\bibitem[{{Loeb} {et~al.}(2005){Loeb}, {Reid}, {Brunthaler}, \&
  {Falcke}}]{Loeb:2005}
{Loeb}, A., {Reid}, M.~J., {Brunthaler}, A., \& {Falcke}, H. 2005, \apj, 633,
  894

\bibitem[{{MacKay}(2003)}]{MacKay:2003}
{MacKay}, D. J.~C. 2003, Information Theory, Inference, and Learning Algorithms
  (Cambridge University Press)

\bibitem[{{Majewski} {et~al.}(2007){Majewski}, {Beaton}, {Patterson},
  {Kalirai}, {Geha}, {Mu{\~n}oz}, {Seigar}, {Guhathakurta}, {Gilbert}, {Rich},
  {Bullock}, \& {Reitzel}}]{Majewski:2007}
{Majewski}, S.~R., {et~al.} 2007, \apjl, 670, L9

\bibitem[{{Martin} {et~al.}(2006){Martin}, {Ibata}, {Irwin}, {Chapman},
  {Lewis}, {Ferguson}, {Tanvir}, \& {McConnachie}}]{Martin:2006}
{Martin}, N.~F., {Ibata}, R.~A., {Irwin}, M.~J., {Chapman}, S., {Lewis}, G.~F.,
  {Ferguson}, A.~M.~N., {Tanvir}, N., \& {McConnachie}, A.~W. 2006, \mnras,
  371, 1983

\bibitem[{{McConnachie} \& {Irwin}(2006)}]{McConnachie:2006}
{McConnachie}, A.~W., \& {Irwin}, M.~J. 2006, \mnras, 365, 902

\bibitem[{{McConnachie} {et~al.}(2005){McConnachie}, {Irwin}, {Ferguson},
  {Ibata}, {Lewis}, \& {Tanvir}}]{McConnachie:2005}
{McConnachie}, A.~W., {Irwin}, M.~J., {Ferguson}, A.~M.~N., {Ibata}, R.~A.,
  {Lewis}, G.~F., \& {Tanvir}, N. 2005, \mnras, 356, 979

\bibitem[{{Metz} {et~al.}(2007){Metz}, {Kroupa}, \& {Jerjen}}]{Metz:2007}
{Metz}, M., {Kroupa}, P., \& {Jerjen}, H. 2007, \mnras, 374, 1125

\bibitem[{O'Hagan \& Leonard(1976)}]{OHagan:1976a}
O'Hagan, A., \& Leonard, T. 1976, Biometrika, 63, 201

\bibitem[{{Prugniel} \& {Simien}(1997)}]{Prugniel:1997}
{Prugniel}, P., \& {Simien}, F. 1997, \aap, 321, 111

\bibitem[{{Rood}(1979)}]{Rood:1979}
{Rood}, H.~J. 1979, \apj, 232, 699

\bibitem[{{Rubin} \& {Ford}(1970)}]{Rubin:1970}
{Rubin}, V.~C., \& {Ford}, Jr., W.~K. 1970, \apj, 159, 379

\bibitem[{{Sales} {et~al.}(2007){Sales}, {Navarro}, {Abadi}, \&
  {Steinmetz}}]{Sales:2007b}
{Sales}, L.~V., {Navarro}, J.~F., {Abadi}, M.~G., \& {Steinmetz}, M. 2007,
  \mnras, 379, 1464

\bibitem[{{Terzi{\'c}} \& {Graham}(2005)}]{Terzic:2005}
{Terzi{\'c}}, B., \& {Graham}, A.~W. 2005, \mnras, 362, 197

\bibitem[{{van den Bergh}(1981)}]{van-den-Bergh:1981}
{van den Bergh}, S. 1981, \pasp, 93, 428

\bibitem[{{van der Marel} \& {Guhathakurta}(2008)}]{vdm}
{van der Marel}, R.~P., \& {Guhathakurta}, P. 2008, \apj, 678, 187

\bibitem[{{Watkins} {et~al.}(2010){Watkins}, {Evans}, \& {An}}]{Watkins:2010}
{Watkins}, L.~L., {Evans}, N.~W., \& {An}, J.~H. 2010, \mnras, 406, 264

\bibitem[{{Widrow} {et~al.}(2008){Widrow}, {Pym}, \& {Dubinski}}]{Widrow:2008}
{Widrow}, L.~M., {Pym}, B., \& {Dubinski}, J. 2008, \apj, 679, 1239

\bibitem[{{York} {et~al.}(2000){York}, {Adelman}, {Anderson}, {Anderson},
  {Annis}, {Bahcall}, {Bakken}, {Barkhouser}, {Bastian}, {Berman}, {Boroski},
  {Bracker}, {Briegel}, {Briggs}, {Brinkmann}, {Brunner}, {Burles}, {Carey},
  {Carr}, {Castander}, {Chen}, {Colestock}, {Connolly}, {Crocker}, {Csabai},
  {Czarapata}, {Davis}, {Doi}, {Dombeck}, {Eisenstein}, {Ellman}, {Elms},
  {Evans}, {Fan}, {Federwitz}, {Fiscelli}, {Friedman}, {Frieman}, {Fukugita},
  {Gillespie}, {Gunn}, {Gurbani}, {de Haas}, {Haldeman}, {Harris}, {Hayes},
  {Heckman}, {Hennessy}, {Hindsley}, {Holm}, {Holmgren}, {Huang}, {Hull},
  {Husby}, {Ichikawa}, {Ichikawa}, {Ivezi{\'c}}, {Kent}, {Kim}, {Kinney},
  {Klaene}, {Kleinman}, {Kleinman}, {Knapp}, {Korienek}, {Kron}, {Kunszt},
  {Lamb}, {Lee}, {Leger}, {Limmongkol}, {Lindenmeyer}, {Long}, {Loomis},
  {Loveday}, {Lucinio}, {Lupton}, {MacKinnon}, {Mannery}, {Mantsch}, {Margon},
  {McGehee}, {McKay}, {Meiksin}, {Merelli}, {Monet}, {Munn}, {Narayanan},
  {Nash}, {Neilsen}, {Neswold}, {Newberg}, {Nichol}, {Nicinski}, {Nonino},
  {Okada}, {Okamura}, {Ostriker}, {Owen}, {Pauls}, {Peoples}, {Peterson},
  {Petravick}, {Pier}, {Pope}, {Pordes}, {Prosapio}, {Rechenmacher}, {Quinn},
  {Richards}, {Richmond}, {Rivetta}, {Rockosi}, {Ruthmansdorfer}, {Sandford},
  {Schlegel}, {Schneider}, {Sekiguchi}, {Sergey}, {Shimasaku}, {Siegmund},
  {Smee}, {Smith}, {Snedden}, {Stone}, {Stoughton}, {Strauss}, {Stubbs},
  {SubbaRao}, {Szalay}, {Szapudi}, {Szokoly}, {Thakar}, {Tremonti}, {Tucker},
  {Uomoto}, {Vanden Berk}, {Vogeley}, {Waddell}, {Wang}, {Watanabe},
  {Weinberg}, {Yanny}, \& {Yasuda}}]{York:2000}
{York}, D.~G., {et~al.} 2000, \aj, 120, 1579

\bibitem[{{Zucker} {et~al.}(2004){Zucker}, {Kniazev}, {Bell},
  {Mart{\'{\i}}nez-Delgado}, {Grebel}, {Rix}, {Rockosi}, {Holtzman},
  {Walterbos}, {Annis}, {York}, {Ivezi{\'c}}, {Brinkmann}, {Brewington},
  {Harvanek}, {Hennessy}, {Kleinman}, {Krzesinski}, {Long}, {Newman}, {Nitta},
  \& {Snedden}}]{Zucker:2004}
{Zucker}, D.~B., {et~al.} 2004, \apjl, 612, L121

\bibitem[{{Zucker} {et~al.}(2007){Zucker}, {Kniazev},
  {Mart{\'{\i}}nez-Delgado}, {Bell}, {Rix}, {Grebel}, {Holtzman}, {Walterbos},
  {Rockosi}, {York}, {Barentine}, {Brewington}, {Brinkmann}, {Harvanek},
  {Kleinman}, {Krzesinski}, {Long}, {Neilsen}, {Nitta}, \&
  {Snedden}}]{Zucker:2007}
---. 2007, \apjl, 659, L21

\end{thebibliography}

% -------- APPENDICES --------

\appendix

\section{The Skew-Normal Distribution}
\label{sect:sn}


The skew-normal density function is given by
\citep{OHagan:1976a,Azzalini:1996}
\begin{equation}
    \eqlabel{sn}
    \tilde{\mathcal{N}} (x | \xi, \omega, \beta) =
        \left ( \frac{2}{\omega} \right )
        \phi \left ( \frac{x-\xi}{\omega} \right )
        \, \Phi \left [ \beta \left ( \frac{x-\xi}{\omega} \right ) \right ],
\end{equation}
where
\begin{equation}
    \phi (x) = \frac{1}{\sqrt{2\pi}} e^{-x^2/2}
\end{equation}
is the standard normal distribution and
\begin{equation}
    \Phi(x) = \int _{-\infty} ^x \phi(x^\prime) \, dx^\prime
        = \frac{1}{2} \left [ 1 +
            \mathrm{erf} \left ( \frac{x}{\sqrt{2}} \right )\right ]
\end{equation}
is the cumulative distribution of $\phi(x)$.  In the limit $\beta \to 0$,
we recover the standard normal distribution with mean $\xi$ and width
$\omega$.  The skew-normal distribution is especially appealing because it
can be efficiently sampled pseudo-randomly. The skew-normal parameterization
that produces a particular set of moments---specifically, the mean, variance
and skew---must be determined numerically but this can be done as part of
the preprocessing of the data so this is not a serious obstacle.


\section{Change of Coordinate Basis}

\label{sect:coords}

To convert from observed right ascension $\alpha$, declination $\delta$ and
distance $r$ to M31-centric coordinates, we start with a position
$\vector{r}$ measured in the equatorial basis $S \equiv \{\ex,\ey,\ez\}$
\begin{equation}
    \vector{r}_S =
    \left ( \begin{array}{c}
        \vector{r} \cdot \ex \\
        \vector{r} \cdot \ey \\
        \vector{r} \cdot \ez
    \end{array} \right )
    = r \,\left ( \begin{array}{c}
        \cos \delta \, \cos \alpha \\
        \cos \delta \, \sin \delta \\
        \sin \delta
    \end{array} \right )
\end{equation}
centered on the Sun.  We then transform to a basis $S^\prime \equiv
\{\epx,\epy,\epz\}$
centered on M31. In the $S^\prime$ basis, the unit vectors are defined with
$\epx$ and $\epy$ in the plane of M31's disk and $\epz$ normal to the plane.
$\epx$ is the projection of our line-of-sight onto the disk. The full
rotation corrects for (a) the orientation of the ``normal triad'' at the
position of M31, (b) the position angle of the stellar disk and (c) the
inclination of the disk with respect to the line-of-sight.

The final transformation from $\vector{r} \in S$ to
$\vector{r}^\prime \in S^\prime$ takes the form
\begin{equation}
    \vector{r}^\prime = \R \, [\vector{r} - \posand]
\end{equation}
where $\matrix{R}$ is a rotation matrix that corrects for the effects
mentioned previously and \posand\ is the position of M31 measured in the
$S$ basis. The full rotation is given by
\begin{equation}
    \eqlabel{fullrot}
    \matrix{R} = \Ry (90^\circ - i) \, \Rx (\theta - 90^\circ)
        \, \Rrpq (\alpha_\mathrm{M31}, \delta_\mathrm{M31})
\end{equation}
where $\R_n$ is the \emph{right-handed} Cartesian rotation matrix about the
given axis $n$ and \Rrpq\ is
\begin{equation}
    \eqlabel{triad}
    \Rrpq (\alpha, \delta) \equiv \left ( \begin{array}{ccc}
    \cos \delta \cos \alpha & \cos \delta \sin \alpha & \sin \delta \\
    -\sin \alpha & \cos \alpha & 0 \\
    -\sin \delta \cos \alpha & -\sin \delta \sin \alpha & \cos \delta
    \end{array} \right ) \quad .
\end{equation}
In \eq{fullrot}, $i = 77.5^\circ$ is M31's inclination and
$\theta = 37.7^\circ$ is the position angle of the stellar disk measured
from north over east \citep{de-Vaucouleurs:1958}.  Numerically, this matrix
is
\begin{equation}
    \eqlabel{fullrotation}
    \R \equiv \left ( \begin{array}{ccc}
0.7752 & 0.3204 & 0.5445 \\
-0.6261 & 0.5042 & 0.5947 \\
-0.0840 & -0.8019 & 0.5915
\end{array} \right )
\end{equation}
which differs from the transformation from \citet{Metz:2007} by a
$z$-rotation of $\sim 0.4^\circ$ since they define $\mathbf{e}_z^\prime$
in the direction of the Galactic center instead of the Sun. For our
purposes, this rotation is unnecessary.

\section{Velocity Corrections}

The kinematics that we observe for the satellite galaxies of M31 must be
corrected for M31's systematic motion if we wish to model the relative
velocity distribution. This is essential for our analysis because the
relative velocity of a satellite is the quantity that traces the underlying
potential. Unfortunately, this systematic motion has only been constrained
along our line-of-site and not in the plane of the sky. This transverse
motion is generally assumed to be negligible but we will show that it is
not difficult marginalize or infer this unknown under the assumptions of
our model.

We begin by writing the full three-dimensional velocity of a given satellite
galaxy $S$ with respect to M31 (labeled $A$) as
\begin{equation}
    \eqlabel{vdm1}
    \vrel{S}{A} = \vrel{S}{\odot} + \vrel{\odot}{A}
\end{equation}
where $\vrel{X,Y} = \vector{v}_{X}-\vector{v}_Y$ is the velocity of $X$
with respect to $Y$.  This equation can be rewritten
\begin{equation}
    \vrel{S}{A} = \vrel{S}{\odot}+\vrel{\odot}{G}+\vrel{G}{\odot}
                        +\vrel{\odot}{A}
               = \vrel{S}{G} + \vrel{G}{A}
\end{equation}
where $G$ indicates the Milky Way.  Then, we decompose $\vrel{G}{A}$
into a component along our line-of-sight to M31 $\los$ and a
component in the plane of the sky $\losperp$
\begin{equation}
    \vrel{S}{A} = \vrel{S}{G} + [\vrel{G}{A}\cdot\los] \, \los
        + [\vrel{G}{A}\cdot\losperp] \, \losperp \quad .
\end{equation}
$\vrel{G}{A}\cdot\los$ has been tightly
constrained \citep{Courteau:1999} but
$\vrel{\odot}{A}\cdot\losperp$ is not currently directly
measurable. Therefore, all measurements of satellite line-of-sight
velocities are contaminated by the transverse velocity term.

\citet{Bahcall:1981}---and subsequently \citet{ew} and \citet{vdm}---pointed
out that the line-of-sight velocities of satellites at large angular
distances will be most significantly affected by this unmeasured systematic
transverse motion and dynamical modeling is highly sensitive to this.
Following \citet{vdm}, we project \eq{vdm1}
onto the line-of-sight of the satellite $\lossat$ we find that the
peculiar velocity of a satellite along the line-of-sight is
\begin{equation}
    \eqlabel{vdm2}
    \vrel{S}{A}\cdot\lossat = \vel{S}\cdot\lossat
                - [ \vel{A}\cdot\los] \, \cos \angsep
                - [ \vel{A}\cdot\losperp] \,
        \sin \Phi_\mathrm{S} \, \cos \left ( \posang
            - \posangperp \right ).
\end{equation}
where $\vel{S}$ and $\vel{A}$ are observed heliocentric velocities of
the satellite galaxy and M31 respectively. In \eq{vdm2},
$\angsep = \cos^{-1} ( \los\cdot\lossat )$ is the angular separation
between the satellite and M31,
$\posang$ is the position angle of the satellite with respect to M31
and $\posangperp$ is the position angle measured from $\losperp$.
These position angles are measured from north through east so that we
can compare this to equation 3 from
\citet{vdm}.

It is clear from \eq{vdm2} that the line-of-sight velocities of
the satellites will exhibit an oscillation as a function of position angle
that will depend on the transverse velocity of M31.  This effect will be
strongest for satellites at large angular separations.  Although it is not
immediately visible in the raw data \citep[see figure 1 of][]{vdm}, this
effect can be used to constrain the transverse velocity. \citet{vdm}
modeled the distribution of satellites as isotropic and isothermal in
order to constrain the systematic transverse motion of M31. In other words,
they approximated $\vrel{S}{A}\cdot\lossat$ as a Gaussian with zero mean
and constant dispersion for each satellite. \citet{vdm} then inferred
$\vel{A}\cdot\losperp$ by fitting \eq{vdm2} to the data using least-squares.

Instead of making a similarly restrictive assumption, we include two extra
parameters in our inference procedure, specifying the two components of M31's
transverse motion. By doing this, we are able to self-consistently model and
infer both the mass profile and transverse motion. Specifically, the
model parameters that we infer are
\begin{equation}
    \begin{array}{r c l}
    \vw & \equiv & [\vel{A}\cdot\losperp]
            \, \cos\left ( \posangperp + \frac{\pi}{2} \right ) \\
    \vn & \equiv & [\vel{A}\cdot\losperp]
            \, \sin \left ( \posangperp + \frac{\pi}{2} \right )
    \end{array}
\end{equation}
since they are directly comparable to the parameters of the same names
from \citet{vdm}.


% -------- MEDIA --------

\clearpage

\begin{figure}[tbp]
    \plotone{plots/m2l.pdf}
    \caption{The constraints on the stellar mass-to-light raitos of the disk
        and bulge with only uninformative priors (run A; green) and with the
        physically motivated prior on the mass-to-light ratio of the disk
        (run B; blue). The contours indicate the
        $0.5\sigma$, $1\sigma$, $1.5\sigma$ and $2\sigma$ levels and the
        grayscale colormap shows the approximate distribution in the central
        regions where the plot is saturated. The shaded magenta region
        approximately shows the range of parameter space excluded by the
        relative overall normalizations of the surface brightness profile
        and the rotation curve.
        The black lines are the levels where the mass-to-light ratios are
        related by constants 1, $\sqrt{2}$ and 2.
        The cyan contours are the constraints on the disk mass-to-light
        ratio from \citet{Bell:2003} and the shaded orange region shows the
        expected value from \citet{Courteau:1999a}.
        \figlabel{m2l}}
\end{figure}

\begin{figure}[tbp]
    \plotone{plots/mass-profile.pdf}
    \caption{Constraints on the mass profile of M31 from run A (green) and
    run B (blue). The orange points are values from the literature (see
    \tab{lit}).
    \figlabel{mass-profile}}
\end{figure}

\begin{figure}[tbp]
    \plotone{plots/m300.pdf}
    \caption{\figlabel{m300}}
\end{figure}

\begin{figure}[tbp]
    \plotone{plots/vt.pdf}
    \caption{The constraints on the transverse motion of M31 on the sky. The
        black points are samples from the marginalized posterior probability
        function of the transverse motion.
        The magenta points are the constraints from \citet{vdm} and the red
        point is the weighted average of those points --- the value that
        \citet{vdm} prefer. The cyan rectangle indicates the approximate
        region excluded theoretically by \citet{Loeb:2005} based on the
        proper motion of M33. The green dashed lines show the axes of the
        covariance ellipse for the samples. \figlabel{vt}}
\end{figure}


% results


% -------- TABLES --------

\clearpage

\begin{deluxetable}{l r rrrr r}
    \tablecaption{Our sample of satellite galaxies. \label{tab:satdata}}
    \tablewidth{0pt}
    \tablehead{
\colhead{Galaxy }   & \colhead{R.A. } & \colhead{Dec.} & \colhead{$D$ } & \colhead{$v_\mathrm{los}$ } & \colhead{Sources} \\
    & \colhead{(J2000.0)    } & \colhead{(J2000.0)}   & \colhead{(kpc)} & \colhead{(km s$^{-1}$) } &
    }
    \startdata

          M33       &   $1^h 33^m 50.9^s$   &   $+30^\circ 39^\prime 36.8^{\prime\prime}$   &   $809 \pm 24$        &   $-180 \pm 1$        &   1,2 \\
          M32       &   $0^h 42^m 41.8^s$   &   $+40^\circ 51^\prime 54.6^{\prime\prime}$   &   $785 \pm 25$        &   $-205 \pm 3$        &   1,2 \\
         IC 10      &   $0^h 20^m 17.3^s$   &   $+59^\circ 18^\prime 13.6^{\prime\prime}$   &   $660 \pm 65$        &   $-344 \pm 5$        &   1 \\
        NGC 205     &   $0^h 40^m 22.1^s$   &   $+41^\circ 41^\prime 7.1^{\prime\prime}$    &   $824 \pm 27$        &   $-244 \pm 3$        &   1,2 \\
        NGC 185     &   $0^h 38^m 58.0^s$   &   $+48^\circ 20^\prime 14.6^{\prime\prime}$   &   $616 \pm 26$        &   $-202 \pm 7$        &   1,2 \\
        IC 1613     &   $1^h 4^m 47.8^s$    &   $+2^\circ 7^\prime 4.0^{\prime\prime}$      &   $715 \pm 35$        &   $-232 \pm 5$        &   1 \\
        NGC 147     &   $0^h 33^m 12.1^s$   &   $+48^\circ 30^\prime 31.5^{\prime\prime}$   &   $675 \pm 27$        &   $-193 \pm 3$        &   1,2 \\
        Pegasus     &   $23^h 28^m 36.2^s$  &   $+14^\circ 44^\prime 34.5^{\prime\prime}$   &   $919 \pm 30$        &   $-182 \pm 2$        &   1,2 \\
         LGS 3      &   $1^h 3^m 55.0^s$    &   $+21^\circ 53^\prime 6.0^{\prime\prime}$    &   $769 \pm 23$        &   $-286 \pm 4$        &   1,2 \\
         And I      &   $0^h 45^m 39.8^s$   &   $+38^\circ 2^\prime 28.0^{\prime\prime}$    &   $745 \pm 24$        &   $-375.8 \pm 1.4$    &   1,2,3 \\
        And II      &   $1^h 16^m 29.8^s$   &   $+33^\circ 25^\prime 9.0^{\prime\prime}$    &   $652 \pm 18$        &   $-193.6 \pm 1.0$    &   1,2,3 \\
        And III     &   $0^h 35^m 33.8^s$   &   $+36^\circ 29^\prime 52.0^{\prime\prime}$   &   $749 \pm 24$        &   $-345.6 \pm 1.8$    &   1,2,3 \\
         And V      &   $1^h 10^m 17.1^s$   &   $+47^\circ 37^\prime 41.0^{\prime\prime}$   &   $774 \pm 28$        &   $-403 \pm 4$        &   1,2 \\
        And VI      &   $23^h 51^m 46.3^s$  &   $+24^\circ 34^\prime 57.0^{\prime\prime}$   &   $783 \pm 25$        &   $-354 \pm 3$        &   1,2 \\
        And VII     &   $23^h 26^m 31.0^s$  &   $+50^\circ 41^\prime 31.0^{\prime\prime}$   &   $763 \pm 35$        &   $-309.4 \pm 2.3$    &   2,3 \\
        And IX      &   $0^h 52^m 51.1^s$   &   $+43^\circ 11^\prime 48.6^{\prime\prime}$   &   $765^{+5}_{-150}$   &   $-207.7 \pm 2.7$    &   4,5 \\
         And X      &   $1^h 6^m 33.7^s$    &   $+44^\circ 48^\prime 15.8^{\prime\prime}$   &   $703 \pm 70$        &   $-163.8 \pm 1.2$    &   3,6 \\
        And XI      &   $0^h 46^m 21.0^s$   &   $+33^\circ 48^\prime 22.0^{\prime\prime}$   &   $760^{+10}_{-150}$  &   $-419.6 \pm 4.4$    &   5,7 \\
        And XII     &   $0^h 47^m 27.0^s$   &   $+34^\circ 22^\prime 29.0^{\prime\prime}$   &   $830^{+170}_{-30}$  &   $-558.4 \pm 3.2$    &   5,7 \\
        And XIII    &   $0^h 51^m 51.0^s$   &   $+33^\circ 0^\prime 16.0^{\prime\prime}$    &   $910^{+30}_{-160}$  &   $-195.0 \pm 8.4$    &   5,7 \\
        And XIV     &   $0^h 51^m 35.0^s$   &   $+29^\circ 41^\prime 49.0^{\prime\prime}$   &   $740 \pm 74$        &   $-481.0 \pm 2.0$    &   3,8 \\
        And XV      &   $1^h 14^m 18.7^s$   &   $+38^\circ 7^\prime 3.0^{\prime\prime}$     &   $630 \pm 60$        &   $-339 \pm 7$        &   9,10 \\
        And XVI     &   $0^h 59^m 29.8^s$   &   $+32^\circ 22^\prime 36.0^{\prime\prime}$   &   $525 \pm 50$        &   $-385 \pm 6$        &   9,10 \\


        \enddata
        \tablecomments{Sources: 1 - \citet{Evans:2000a}, 2 - \citet{McConnachie:2006}, 3 - \citet{Kalirai:2010}, 4 - \citet{Zucker:2004}, 5 - \citet{Collins:2010}, 6 - \citet{Zucker:2007}, 7 - \citet{Martin:2006}, 8 - \citet{Majewski:2007}, 9 - \citet{Ibata:2007}, 10 - \citet{Letarte:2009}}

\end{deluxetable}


\begin{deluxetable}{cccc}
    \tablecaption{Skew-normal parameterization of distance measurements \label{tab:sn}}
    \tablewidth{0pt}
    \tablehead{
    \colhead{Galaxy} & \colhead{Location ($\xi$)} & \colhead{Scale ($\omega$)} & \colhead{Shape ($\beta$)}
    }
    \startdata

        And IX      &   730     &   115     &   -7  \\
        And XI      &   717     &   107     &   -4  \\
        And XII     &   889     &   109     &   2   \\
        And XIII    &   854     &   103     &   -2  \\

    \enddata
\end{deluxetable}


\begin{deluxetable}{lll}
    \tabletypesize{\footnotesize}
    \tablecaption{Model parameters and their prior distributions \label{tab:params}}
    \tablewidth{0pt}
    \tablehead{\colhead{Parameter} & \colhead{Description} & \colhead{Prior}}
    \startdata


    $\alpha_\mathrm{M31}$                                   &   Right ascension (J2000.0)\tablenotemark{a}  & $00^\mathrm{h} 42^\mathrm{m} 44.4^\mathrm{s}$\\
    $\delta_\mathrm{M31}$                                   &   Declination (J2000.0)\tablenotemark{a}      & $+41^\circ 16^\prime 08^{\prime\prime}$\\
    $\theta$                                                &   Position Angle\tablenotemark{a}             & $77.5^\circ$ \\
    $i$                                                     &   Inclination\tablenotemark{a}                & $37.7^\circ$ \\

    \tableline

    $D_\mathrm{M31}/\mathrm{kpc}$                           &   Distance to M31\tablenotemark{b}            & $\mathcal{N} (785, 25^2)$ \\
    $v_\mathrm{M31,los}/\mathrm{km\,s}^{-1}$                &   Line-of-sight velocity\tablenotemark{c}     & $\mathcal{N} (301, 1)$ \\
    $(v_\mathrm{W},v_\mathrm{N})/\mathrm{km\,s}^{-1}$       &   Systemic transverse velocity                & $\sqrt{v_\mathrm{W}^2+v_\mathrm{N}^2} < 1000$ \\

    \tableline

    $\log_{10} \, a_\mathrm{h}/\mathrm{kpc}$                &   Halo scale length                           & $U(0,3)$  \\
    $\log_{10} \, v_\mathrm{h}/\mathrm{km\,s}^{-1}$         &   Halo characteristic velocity                & $U(0,4)$ \\
    $\log_{10} \, M_\mathrm{d}/10^{9} \, M_\odot$           &   Disk mass scale                             & $U(0.4,2.4)$  \\
    $\log_{10} \, R_\mathrm{d}/\mathrm{kpc}$                &   Disk scale length                           & $U(0,1)$  \\
    $\Upsilon_\mathrm{d}$                                   &   Disk $R$-band mass-to-light ratio           & $U(0,10)$  \\

    $\log_{10} \, v_\mathrm{b}/\mathrm{km\,s}^{-1}$         &   Bulge characteristic velocity               & $U(1,3)$ \\
    $\log_{10} \, R_\mathrm{e}/\mathrm{kpc}$                &   Bulge scale length                          & $U(-2,0.9)$  \\
    $n$                                                     &   Bulge S\'ersic index                        & $2$  \\
    $\Upsilon_\mathrm{b}$                                   &   Bulge $R$-band mass-to-light ratio          & $U(0,10)$  \\

    \tableline

    $\log_{10}\,\sqrt{\epsilon_\mathrm{RC}}/\mathrm{km\,s}^{-1}$   &   Rotation curve noise parameter              & $U(-1,6)$  \\
    $\log_{10}\,\sqrt{\epsilon_\mathrm{b}}/\mathrm{km\,s}^{-1}$    &   Bulge velocity dispersion noise parameter   & $U(-12,12)$  \\
    $\log_{10}\,\sqrt{\epsilon_\mu}/\mathrm{mag\,arcsec}^{-2}$                         &   Surface brightness profile noise parameter  & $U(-8,6)$  \\


    \enddata

    \tablenotetext{a}{\citet{de-Vaucouleurs:1958,de-Vaucouleurs:1991}}
    \tablenotetext{b}{\citet{McConnachie:2005}}
    \tablenotetext{c}{\citet{Courteau:1999}}
\end{deluxetable}




\begin{deluxetable}{lccc}
    % \tabletypesize{\footnotesize}
    \tablecaption{Model parameters and their prior distributions \label{tab:results}}
    \tablewidth{0pt}
    \tablehead{\colhead{Parameter} & \colhead{All} & \colhead{No fast} & \colhead{Stability}}
    \startdata


    $D_\mathrm{M31}/\mathrm{kpc}$                                   &    $755^{+16}_{-18}$       & $766^{+17}_{-18}$       & $765 \pm 16$           \\
    $v_\mathrm{W}/\mathrm{km\,s}^{-1}$                              &    $28 \pm 248$            & $12 \pm 214$            & $27 \pm 151$           \\
    $v_\mathrm{N}/\mathrm{km\,s}^{-1}$                              &    $58 \pm 126$            & $-15 \pm 116$           & $-7 \pm 76$            \\
    $\rho_\mathrm{WN}$                                              &    0.238                   & 0.101                   & 0.222                  \\\tableline
    $\log_{10} \, a_\mathrm{h}/\mathrm{kpc}$                        &    $1.80^{+0.14}_{-0.12}$  & $1.73^{+0.15}_{-0.18}$  & $0.91^{+0.26}_{-0.11}$ \\
    $\log_{10} \, v_\mathrm{h}/\mathrm{km\,s}^{-1}$                 &    $2.63 \pm 0.03$         & $2.62^{+0.04}_{-0.03}$  & $2.68^{+0.03}_{-0.02}$ \\
    $\log_{10} \, M_\mathrm{d}/10^{9} \, M_\odot$                   &    $2.16^{+0.04}_{-0.03}$  & $2.15^{+0.05}_{-0.03}$  & $1.81^{+0.03}_{-0.05}$ \\
    $\log_{10} \, R_\mathrm{d}/\mathrm{kpc}$                        &    $0.69^{+0.04}_{-0.03}$  & $0.68^{+0.05}_{-0.02}$  & $0.74 \pm 0.01$        \\
    $\Upsilon_\mathrm{d}$                                           &    $5.42^{+1.31}_{-1.26}$  & $4.53^{+1.77}_{-0.54}$  & $2.80^{+0.22}_{-0.29}$ \\
    $\log_{10} \, v_\mathrm{b}/\mathrm{km\,s}^{-1}$                 &    $2.28^{+0.23}_{-0.63}$  & $2.16^{+0.29}_{-0.67}$  & $2.66^{+0.06}_{-0.09}$ \\
    $\log_{10} \, R_\mathrm{e}/\mathrm{kpc}$                        &    $0.00^{+0.21}_{-0.30}$  & $-0.05^{+0.17}_{-0.47}$ & $0.07^{+0.06}_{-0.04}$ \\
    $\Upsilon_\mathrm{b}$                                           &    $0.90^{+5.88}_{-0.47}$  & $1.17^{+6.62}_{-0.77}$  & $1.77^{+0.74}_{-0.59}$ \\\tableline
    $\log_{10}\,\sqrt{\epsilon_\mathrm{RC}}/\mathrm{km\,s}^{-1}$    &    $1.39 \pm 0.03$         & $1.39 \pm 0.03$         & $1.49^{+0.08}_{-0.04}$ \\
    $\log_{10}\,\sqrt{\epsilon_\mathrm{b}}/\mathrm{km\,s}^{-1}$     &    $2.19^{+0.55}_{-0.37}$  & $2.16^{+0.43}_{-0.33}$  & $1.33^{+0.77}_{-1.22}$ \\
    $\log_{10}\,\sqrt{\epsilon_\mu}/\mathrm{mag\,arcsec}^{-2}$      &    $-0.70^{+0.22}_{-0.60}$ & $-0.59^{+0.10}_{-0.43}$ & $-1.24^{+0.26}_{-0.10}$\\


    \enddata
\end{deluxetable}

\begin{deluxetable}{l l c c c}
    \tablecaption{Mass estimates for M31 from the literature \tablabel{lit}}
    \tablewidth{0pt}
    \tablehead{
        \colhead{Source}           &
        \colhead{Method}           &
        \colhead{$D_\mathrm{M31}$} &
        \colhead{$r$}              &
        \colhead{$M (<r)$}         \\
        & &
        \colhead{(kpc)}            &
        \colhead{(kpc)}            &
        \colhead{($10^{11} \, M_\odot$)} \\
    }

\startdata
\citet{Rubin:1970}         & Rotation curve     & 690 & 24       & $1.85 \pm 0.10$ \\
\citet{Gottesman:1970}     & Rotation curve     & --- & $\sim34$ & $\sim 2.2$ \\
\citet{Hartwick:1974}      & Globular clusters  & 667 & 17       & $3.4 \pm 1.4$ \\
\citet{Deharveng:1975}     & Rotation curve     & 690 & 20       & $1.63 \pm 0.15$ \\
\citet{Rood:1979}          & Satellite galaxies & 667 & Total    & $2.45 \pm 0.35$ \\
\citet{Bahcall:1981}       & Satellite galaxies & 667 & 100      & $12.7 \pm 8.0$ \\
\citet{van-den-Bergh:1981} & Satellite galaxies & 667 & Total    & $7.5 \pm 3.9$ \\
                           & Globular clusters  & 667 & 6        & $0.9 \pm 0.2$ \\
                           & Globular clusters  & 667 & 19       & $2.4 \pm 1.2$ \\
\citet{Braun:1991}         & Rotation curve     & 690 & 28       & $2.0 \pm 0.1$\\
\citet{Courteau:1999}      & Satellite galaxies & 760 & Total    & $13.3 \pm 1.8$ \\
\citet{ew}                 & Planetary nebulae  & 770 & $\sim$31 & $\sim2.8$ \\
                           & Globular clusters  & 770 & $\sim$40 & $\sim4.7$ \\
                           & Satellite galaxies & 770 & Total    & $\sim12.3$ \\
\citet{Evans:2000a}        & Satellite galaxies & 770 & Total    & $8.5 \pm 1.5$ \\
\citet{Klypin:2002}        & Rotation curve     & 770 & 100      & $8.3 \pm 0.2$ \\
\citet{Evans:2003}         & Globular clusters  & 770 & 100      & $\sim 12$ \\
\citet{Lee:2008}           & Globular clusters  & 780 & 55       & $5.5^{+0.4}_{-0.3}$ \\
                           & Globular clusters  & 780 & 100      & $19.3 ^{+1.4}_{-1.3}$ \\
\citet{Chemin:2009}        & Rotation curve     & 785 & 38       & $4.7 \pm 0.5$ \\
\citet{Watkins:2010}       & Satellite galaxies & 785 & 100      & $2.1 \pm 1$ \\
                           & Satellite galaxies & 785 & 200      & $12.4 \pm 3.8$ \\
                           & Satellite galaxies & 785 & 300      & $14.0 \pm 4.0$ \\
\enddata
\end{deluxetable}




\end{document}
